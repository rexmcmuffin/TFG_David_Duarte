\documentclass[12pt]{article}

\usepackage{graphicx}
\usepackage{geometry}
\usepackage{setspace}
\usepackage{multirow}
\usepackage[table,xcdraw]{xcolor}
\usepackage{caption}
\usepackage{tocbibind} %Index of table of contents
\usepackage{url} 
\geometry{a4paper, total={170mm, 257mm}, left=20mm, right=20mm, top=25mm, bottom=25mm}
\usepackage[spanish]{babel}

\usepackage{fancyhdr}
\pagestyle{fancy}
\fancyhf{} % Limpia los estilos anteriores de encabezado y pie de página

% Define el contenido del encabezado
\rhead{Tecnológico de Costa Rica}

% Opcional: Puedes personalizar el pie de página si lo necesitas
% \lfoot{Pie de página izquierdo}
\cfoot{\thepage} % Número de página en el centro
% \rfoot{Pie de página derecho}

% Línea horizontal en la parte superior del encabezado
\renewcommand{\headrulewidth}{0.5pt}

\setlength{\parindent}{1em}
\setlength{\parskip}{1em}

\usepackage{makeidx}
\makeindex


%%%%%%%%%%%%%%%%%%%%%%%%%%%%%%%%%%%%%%%%%%%%%%%%%%%%%%%%%%%%%%%%%%%%%%%%%%%%%%%%%%%%%%%%%%%%%%%%%%%%%%%%%%%%%%%%%%%%%% here Begins the document

\begin{document}
\begin{titlepage}

\centering


\vspace{10cm}

\textbf{\LARGE Instituto Tecnológico de Costa Rica}

\vspace{2cm}

\textbf{\LARGE Escuela de Ingeniería Electrónica}

\vspace{2cm}

\includegraphics[width=10cm]{logotec/image.png}
\vspace{2cm}

\hrule

\vspace{1cm}

\textbf{\LARGE Diseño de un sistema para soportar algoritmos de control basados en aprendizaje automático en tiempo real para una planta prototipo de control automático}

\vspace{1cm}

\hrule

\vspace{1cm}

\textbf{\LARGE SIPLab-TEC}

\vspace{1cm}

\textbf{\LARGE David Felipe Duarte Sánchez}

\vspace{1cm}

\textbf{\LARGE 2017239606}

\vspace{1cm}

\today % Esto agrega la fecha actual

\end{titlepage}
.
\par
\vspace{16cm} % Ajusta la cantidad de espacio vertical según sea necesario

Yo, David Duarte Sanchez portador de la cédula 305070982, declaro que los resultados obtenidos en el presente trabajo de investigación, previo a la obtención del titulo de Licenciado en Ingeniería en Electrónica, son absolutamente originales, auténticos y personales.

Soy consciente de que el hecho de no respetar los derechos de autor y realizar una mala conducta científica; es decir, fabricación de datos falsos y plagio, conlleva
sanciones universitarias y/o legales.

En tal virtud, declaro que el trabajo de investigación realizado sujeto a evaluación no ha sido presentado anteriormente para obtener algún grado académico o título, ni ha sido publicado en sitio alguno y los efectos legales y académicos que se puedan derivar del trabajo propuesto de investigación son y serán de mi sola y exclusiva responsabilidad legal y académica.

\newpage
%\renewcommand{\contentsname}{Contenidos} commenting due babel integration
\tableofcontents

\newpage

\section{Entorno del proyecto}

El laboratorio de procesamiento de señales e imágenes SIPLab de la Escuela de Electrónica, enfocado en la solución de problemas de ámbito nacional y regional relacionados con el procesamiento y reconocimiento de información, transportada en señales temporales y espaciales. De momento se trabaja en un proyecto de aplicación del Aprendizaje Automático en tareas de control automático  \cite{SIPLab} \cite{13_se}.

Los sistemas de control han asumido un papel importante en el desarrollo y avance de la civilización moderna y la tecnologia\cite{Elescano}; Ya que el avance tecnológico en las fabricas se hace notar por medio de la implementacion de sistemas orientados al control de calidad de los productos manuafcturados, lineas de ensamble automaticas o bien control de maquinas o herramientas. También se pueden observar en sistemas de control de armas y sistemas espaciales \cite{kuo1996sistemas}. En esta área se han demostrado múltiples aplicaciones en temas como lo pueden ser los sistemas de vehículos automotrices, además de estar presente también en los procesos de modelos de fabricación y en cualquier tipo de operación industrial que requiera control sobre variables físicas como lo son el control de temperatura, presión, humedad o flujo. Esto conlleva a promover la tecnología en áreas como la domótica , ingeniería mecánica, no sin antes mencionar el desarrollo que ha permitido en áreas como lo son la aplicación de técnicas modernas de control las cuales contribuyen al desarrollo de avances científicos y tecnológicos \cite{Corvacho} \cite{Perez}.

%%Incluir una imagen chiva aca

Por un lado, el control automático juega un papel de vital importancia en cuanto al avance de la ciencia e ingeniería, en donde los sistemas de estudio son dinámicos y el conocimiento de la teoría de control nos da acceso una base para la comprensión del funcionamiento de dichos sistemas. Las etapas presentes en el análisis de un sistema de control requieren la intervención manual, donde por medio de la experimentación se deben de ajustar los parámetros, la validez de los mismos es directamente proporcional a que tan precisos fueron los modelos utilizados previamente. Como se observa en la Figura \ref{fig:diagrama_control} se muestran las etapas para el diseño convencional de un sistema de control, en donde los pasos 5, 6 y 7 son procesos iterativos ya que los mismos se pueden mejorar conforme avanza el proceso. 


\begin{figure}[!ht]
  \centering
  \includegraphics[scale=0.5]{diagramas/diagrama_control.png}
  \caption{Proceso de diseño para el control de un sistema Fuente: \cite{Corvacho}}
  \label{fig:diagrama_control}
\end{figure}


Por otro lado los sistemas en tiempo real son definidos como los sistemas informaticos en los que la respuesta de la aplicacion ante estimulos externos debe de realizarce dentro de un plazo de tiempo establecido. La predicibilidad del tiempo de respuesta determina si el sistema sera capaz de ofrecer una respuesta correcta ante la llegada de un estimulo en un tiempo acotado, esto permite conocer a priori cual va a ser el comportamiento del sistema en las peores condiciones y de esta forma realizar un análisis de tiempos de respuesta del sistema \cite{alonso2010panoramica}. En la Figura \ref{fig:diagrama_tiempo_real}, se observa un diagrama descriptivo de la ejecucion de una tarea en un sistema de tiempo real. En donde se observan atributos temporales como: activacion y plazo de respuesta\cite{de2000introduccion}. 


\begin{figure}[!ht]
  \centering
  \includegraphics[scale=0.5]{diagramas/tiempo_real.png}
  \caption{Ejecución de una tarea de tiempo real Fuente: \cite{de2000introduccion}}
  \label{fig:diagrama_tiempo_real}
\end{figure}

\newpage

\section{Definición del problema}

\subsection{Generalidades}

%Describe la necesidad de un sistema de control eficiente en el entorno actual.
Los sistemas de control siempre dependen de un proceso de caracterización del proceso físico, estos se basan en un modelo matemático dependiente del tiempo, este representa la dinámica de la planta. El sistema debe de contar con al menos un actuador y un sensor con el fin de leer datos y realimentar el sistema. Eso se puede obtener de forma teórica mediante el modelado de la planta o bien de forma empírica, en la cual se toman los datos de forma directa de la planta y la misma se somete a un impulso de entrada conocido, con el fin de lograr determinar el comportamiento de la planta, una vez concluido se continua con el estudio de los datos de entrada y salida, el proceso es conocido como identificación. Una de las ventajas de la forma empírica es que mediante la toma de datos de la planta se pueden ingresar pequeñas perturbaciones manuales a la planta en cuestión con el objetivo de observar el comportamiento de la misma \cite{15_tec}.

Con la presencia de las no linealidades en algunos sistemas de control y que no todos los sistemas son compuestos por una entrada y una salida se vuelve mas complejo la descripción del sistema  de control y por ende se requiere una mayor cantidad de ecuaciones, haciendo que cada vez según avancen los sistemas de control los mismo sean mas difíciles de derivar para que los sistemas logren un equilibrio aceptable entre precisión y velocidad de respuesta, además que sin importar que existan métodos clásicos estos no siempre van a conducir a resultados aceptados. Esto dificulta el procesos e obtención de un modelo matemático representativo para el sistema. Ahora bien la mayor parte de sistemas que se encuentran en la naturaleza son no lineales y los sistemas que son aplicados a solucione modernas de ingeniera tienden a presentar problemas de control , tales como los sistemas de comando de vuelo que presentan como mínimo 6 entradas y 4 salidas, es acá donde se torna de gran importancia un método o técnica de control el cual permita identificar y controlar el sistema \cite{15_tec}. 

%Explica el propósito de adquisicion de datos en tiempo real

Por otro lado la demanda de plataformas en tiempo real RT para dichos sistemas aumentan cada año, lo cual lleva a trasladar todas estas aplicaciones a SoC de alto rendimiento tomando en cuenta que un manejo eficiente del entorno en tiempo real tiene la capacidad de acelerar las diversas tareas de control a nivel de software. También se debe de tomar en cuenta que, resultando en la consideracion de las restricciones temporales para ejecutar las actividades en tiempo real con garantia y asegurar que el sistema operativo tiene un coportamiento predecible. Para garantizar el cumplimento de las restricciones es necesario realizar un analisis de planificabilidad que permita conocer si las restricciones impuestas se van a cumplir en cualquier situación. Esto se realiza en funcion de comoy cuando se toman las decisiones de planificación. Se dividen en dos grupos, Off-Line lo que indica que las decisiones se tomaron en la etapa de diseño y On Line cuando las decisiones se toman durante la ejecución.



\subsection{Síntesis del problema}

En la Escuela de Ingeniería Electrónica y el SIPLab, se ha identificado una carencia a nivel de implementación de un sistema de control automático, el cual funcione de la misma forma independientemente del  hardware al cual se conecte y que además de esto simplifique la configuración de parámetros de operación deseados, ya que actualmente limita los procesos de enseñanza tanto en las ares de aprendizaje automático como en control.


\section{Enfoque de la solución}

El enfoque la solución se conduce hacia la implementación de una plataforma de trabajo para un sistema embebido que logre ejecutar en tiempo real un sistema de control automatico neuronal.

%Como se menciono en el apartado de definición del problema la implementación de un software agnóstico de hardware como interfaz de comunicación entre el usuario y la planta a controlar presenta la eliminación de muchas dependencias además de permitir una portabilidad del controlador a cualquier plataforma de hardware que posea entradas y salidas programables. Es por esto que a continuación se proponen tres alternativas de implementación para el control de una planta prototipo no lineal en todo su espacio de operación mediante técnicas de control automático que el SIPLab ha investigado en el ultimo año. La planta seleccionada para la implementación del esta interfaz sera la planta de péndulo amortiguado a hélice (PAHM). Además de esto se propone utilizar un sistema embebido de altas prestaciones que sea capaz de ejecutar el programa, lo que a su vez impone limitaciones a la complejidad del algoritmo a seleccionar.

Finalmente, basándonos en matriz de Pugh se analizan las soluciones con el objetivo de seleccionar la solución mas adecuada, en donde un cero indica que la alternativa es similar, un +1 que la alternativa es mejor y un -1 que la alternativa es peor en el criterio comparada a la referencia.

\subsection{Alternativa 1}

Como primer alternativa de solución se propone hacer uso de técnicas de control adaptativo, técnica que da sus comienzos en la década de los 80, aunque existían limitaciones tecnológicas que hacían que esta alternativa fuera muy costosa. Actualmente,
no está sujeto a esas limitaciones, pues se puede desarrollar esta técnica a bajo costo y con un procesamiento alto y rápido\cite{15_tec}.

El control adaptativo busca mejorar el funcionamiento de la planta modificando su comportamiento en respuesta a los cambios en la dinámica del sistema y a perturbaciones externas, que con el tiempo deterioran su funcionamiento. Además, este permite realizar ajustes al controlador en tiempo real. Esto se realiza utilizando técnicas que miden las variables dinámicas de la planta de forma continua, las compara con parámetros deseados y mediante su diferencia modifica las características del controlador, generando así un accionamiento que mantiene las variables de la planta en un rango de desempeño\cite{15_tec}.

Se haría uso de un filtro de Kalman para la estimación de parámetros o variables de estado cuando el sistema presenta ruidos aditivos. Además, este filtro proporciona una predicción del estado futuro del sistema, basado en estimaciones pasadas, lo que ayuda al controlador adaptativo a adaptarse de mejor forma ante problemas o perturbaciones que ocurran en el sistema\cite{15_tec}.

\subsection{Alternativa 2}

Como segunda alternativa se propone hacer uso de aprendizaje automático para construir una plataforma de trabajo para un sistema embebido que logre ejecutar en tiempo real un sistema de control automatico neuronal, esto con el objetivo de llegar a establecer control de la planta PAHM. Esto se logra mediante el uso de redes neuronales artificiales ya que una de las principales características de estas mismas es el aprendizaje de relaciones complejas a partir de un conjunto de ejemplos. Además de esto debido a sus capacidades de aproximación y adaptabilidad las RNA presentan una muy buena alternativa en el modelado de sistemas no lineales y en la comprensión de los mismos. Como una alternativa a los controladores clásicos es ofrecido en el caso del aprendizaje reforzado RL, el cual ofrece algunos algoritmos para el desarrollo de controladores óptimos de sistemas con dinámicas no lineales\cite{15_tec}. 

Estos algoritmos mencionados anteriormente operan bajo la metodología la cual tiene como objetivo que un agente sea capaz de encontrar la acción correcta de manera autónoma, explorando un espacio desconocido y determinando la acción mediante prueba y error ya que estos sistemas aprenden por medio de la aplicación de recompensas y penalizaciones las cuales obtienen mediante sus acciones, esto con el objetivo de crear la mejor estrategia posible. Cabe destacar que esta metodología se ha aplicado en otros sistemas de control en donde no basta con la aplicación de las técnicas de control clásicas. Permitiendo de esta forma automatizar el proceso mas allá de lo alcanzable con los métodos tradicionales\cite{naizhang2022intel}. 


\subsection{Alternativa 3}

La tercera alternativa de solución es similar a lo planteado en la anterior, ya que también hace uso de una red neuronal artificial basada en aprendizaje reforzado para controlar la planta PAHM, debido a su facilidad de aproximación, su adaptabilidad y demás características anteriormente expuestas. Lo que hace diferente a esta alternativa, es la aplicación de una única red neuronal directamente sobre la planta PAHM\cite{15_tec}.

\subsection{Selección de la solución}

Una vez contempladas las bondades que ofrecen las 3 alternativas el criterio de desciño se basa en los resultados obtenidos mediante la generación de la matriz de Pugh la cual se puede observar en la Tabla \ref{tab:pugh}.

% Please add the following required packages to your document preamble:
% \usepackage[table,xcdraw]{xcolor}
% Beamer presentation requires \usepackage{colortbl} instead of \usepackage[table,xcdraw]{xcolor}
\begin{table}[!h]
  \centering
  \caption{Matriz de Pugh para la selección de la solución}
  \label{tab:pugh}
  \begin{tabular}{lllll}
  \cline{2-5}
  \multicolumn{1}{l|}{}                                                       & \multicolumn{1}{l|}{\cellcolor[HTML]{DAE8FC}Peso}   & \multicolumn{1}{l|}{\cellcolor[HTML]{DAE8FC}Alternativa 1} & \multicolumn{1}{l|}{\cellcolor[HTML]{DAE8FC}Alternativa 2} & \multicolumn{1}{l|}{\cellcolor[HTML]{DAE8FC}Alternativa 3} \\ \hline
  \multicolumn{1}{|l|}{\cellcolor[HTML]{DAE8FC}Criterios}                     & \multicolumn{1}{l|}{\cellcolor[HTML]{CBCEFB}}      & \multicolumn{1}{l|}{} & \multicolumn{1}{l|}{} & \multicolumn{1}{l|}{} \\ \hline
  \multicolumn{1}{|l|}{\cellcolor[HTML]{DAE8FC}Ejecución en tiempo real}      & \multicolumn{1}{l|}{\cellcolor[HTML]{CBCEFB}5}      & \multicolumn{1}{l|}{$=$} & \multicolumn{1}{l|}{0} & \multicolumn{1}{l|}{0} \\ \hline
  \multicolumn{1}{|l|}{\cellcolor[HTML]{DAE8FC}Seguridad e integridad de la planta} & \multicolumn{1}{l|}{\cellcolor[HTML]{CBCEFB}5} & \multicolumn{1}{l|}{$=$} & \multicolumn{1}{l|}{0} & \multicolumn{1}{l|}{$-1$} \\ \hline
  \multicolumn{1}{|l|}{\cellcolor[HTML]{DAE8FC}Porteabilidad de la solución}  & \multicolumn{1}{l|}{\cellcolor[HTML]{CBCEFB}5}      & \multicolumn{1}{l|}{$=$} & \multicolumn{1}{l|}{$+1$} & \multicolumn{1}{l|}{$-1$} \\ \hline
  \multicolumn{1}{|l|}{\cellcolor[HTML]{DAE8FC}Ajuste de parametros}          & \multicolumn{1}{l|}{\cellcolor[HTML]{CBCEFB}3}      & \multicolumn{1}{l|}{$=$} & \multicolumn{1}{l|}{$+1$} & \multicolumn{1}{l|}{$+1$} \\ \hline
  \multicolumn{1}{|l|}{\cellcolor[HTML]{DAE8FC}Tiempo de desarrollo}          & \multicolumn{1}{l|}{\cellcolor[HTML]{CBCEFB}2}      & \multicolumn{1}{l|}{$=$} & \multicolumn{1}{l|}{$-1$} & \multicolumn{1}{l|}{$+1$} \\ \hline
  \multicolumn{1}{|l|}{\cellcolor[HTML]{DAE8FC}Coste Económico}               & \multicolumn{1}{l|}{\cellcolor[HTML]{CBCEFB}4}      & \multicolumn{1}{l|}{$=$} & \multicolumn{1}{l|}{$+1$} & \multicolumn{1}{l|}{$-1$} \\ \hline
   &  &  &  &  \\ \cline{1-1} \cline{3-5} 
  \multicolumn{1}{|l|}{\cellcolor[HTML]{DAE8FC}Suma General} & \multicolumn{1}{l|}{} & \multicolumn{1}{l|}{0} & \multicolumn{1}{l|}{10} & \multicolumn{1}{l|}{$-9$} \\ \cline{1-1} \cline{3-5} 
  \multicolumn{1}{|l|}{\cellcolor[HTML]{DAE8FC}Posición} & \multicolumn{1}{l|}{} & \multicolumn{1}{l|}{2} & \multicolumn{1}{l|}{1} & \multicolumn{1}{l|}{3} \\ \cline{1-1} \cline{3-5} 
  \end{tabular}
  \end{table}


Cuando se realiza el análisis de la Matriz de Pugh se puede determinar que en aspectos como lo son ajustes de parámetros o innovación la Alternativa 2 y 3 se encuentran empatadas ya que hay que recordar que lo que difiere entre estas dos opciones es el método de implementación, por un lado en la Alternativa 2 se propone el uso de un software agnóstico de Hardware como base del diseño mientras que en la Alternativa 3 se plantea el uso de un método mas tradicional y dependiente del hardware, eliminando virtudes de la alternativa 2 como lo era la portabilidad de la solución a otros sistemas de control sin importar sobre el hardware que se ejecutara el mismo. Es por esto que criterios como lo fueron el costo económico terminaron de inclinar la balanza por el uso de la Alternativa 2. 

\section{Meta}

Analizar y procesar los datos en tiempo real mediante el diseño de una plataforma de trabajo para un sistema embebido para que de esta forma se logre ejecutar en tiempo real un sistema de control automatico neuronal.

%Controlar exitosamente la planta prototipo utilizando aprendizaje reforzado, estabilizando el sistema en un rango de tiempo máximo del 10 \% del control clásico, con un sobre impulso inferior al 5 \%, cero error de estado estacionario y la reducción de perturbaciones de entrada o salida a la planta.

\section{Objetivo General}

Diseñar una plataforma de trabajo para un sistema embebido que logre ejecutar en tiempo real un sistema de control automatico neuronal.

\textbf{Indicador:} Cumplir al menos un 90 \% de las metricas especificadas en la Tabla \ref{tab:obj_1}.\newline
\textbf{Entregable:} .


\begin{table}[!h]
  \centering
    \caption{Requerimientos para satisfacer el indicador del Objetivo general}
    \label{tab:obj_1}
    \begin{tabular}{|l|l|}
    \hline
    Id & Requerimientos \\ \hline
    C-01 & Latencia \\ \hline
    C-02 & Tasa de éxito de transmisión \\ \hline
    C-03 & Disponibilidada \\ \hline
    C-04 & Escalabilidad del sistema \\ \hline
    C-05 & Confiabilidad \\ \hline
    C-06 & Presición de los datos \\ \hline
    C-07 & Saturacion del sistema \\ \hline
    C-08 & Tiempo de procesamiento \\ \hline
    C-09 & Jitter \\ \hline
    \end{tabular}
    \end{table}

\section{Objetivos Específicos}

\begin{enumerate}
    \item Investigar acerca de plataformas que garantizen que el análisis de los datos se de en tiempo real. \newline
    \textbf{Indicador:} Se genera una matriz de pugh con al menos 6 criterios.\newline
    \textbf{Entregable:} Matriz de pugh con al menos 6 criterios a evaluar del sistema en tiempo real y el sistema seleccionado.
    \item Implementar la plataforma seleccionada \newline
    \textbf{Indicador:} Se hace la implementación y se verifica que la misma cumpla con los requerimientos de la Tabla \ref{tab:obj_1}.\newline
    \textbf{Entregable:} El sistema implementado.
    \item Evaluar la plataforma seleccionada \newline
    \textbf{Indicador:} Se evaluan al menos 3 configuraciones del sistema y se utilizan al menos 3 métricas para la evaluación del desempeño de las mismas.\newline
    \textbf{Entregable:} Métricas de calidad en donde se define cual confuiguración es mejor según la métrica seleccionada.
\end{enumerate}

\section{Procedimiento para la ejecución del proyecto}

Sobre la ejecución del proyecto se planifican las actividades de forma jerárquica, en donde se establecen actividades según los objetivos planteados, los requisitos de las actividades y la dependencia que tenga este mismo con alguna actividad previa. Además de esto se definen los tiempos de duración de las actividades las cuales se resumen en la Tabla \ref{tab:actividades}.

% Please add the following required packages to your document preamble:
% \usepackage{multirow}
\begin{table}[ht]
  \centering
  \caption{Procedimiento para la ejecución del proyecto}
  \label{tab:actividades}
  \begin{tabular}{|c|l|l|l|}
  \hline
  Objetivo                                                                                                                                                                  & Actividad                                                                                                                      & Tiempo en días & Dependencias \\ \hline
  \multirow{4}{*}{\begin{tabular}[c]{@{}c@{}}1.Investigar acerca de \\ plataformas que \\ garantizen que el análisis \\ de los datos \\ se de en tiempo real.\end{tabular}} & \begin{tabular}[c]{@{}l@{}}1.1 Investigar sobre \\ sistemas en tiempo \\ real.\end{tabular}                                    & 10             &              \\ \cline{2-4} 
                                                                                                                                                                            & \begin{tabular}[c]{@{}l@{}}1.2 Investigar sobre \\ análisis de datos en \\ tiempo real.\end{tabular}                           & 10             & 1.1          \\ \cline{2-4} 
                                                                                                                                                                            & \begin{tabular}[c]{@{}l@{}}1.3 Definir los criterios \\ del sistema de tiempo \\ real.\end{tabular}                            & 5              & 1.2          \\ \cline{2-4} 
                                                                                                                                                                            & \begin{tabular}[c]{@{}l@{}}1.4 Elección del ranking \\ de sistemas.\end{tabular}                                               & 8              & 1.3          \\ \hline
  \multicolumn{1}{|l|}{\multirow{2}{*}{\begin{tabular}[c]{@{}l@{}}2. Implementar la plataforma \\ seleccionada.\end{tabular}}}                                              & \begin{tabular}[c]{@{}l@{}}2.1 Implementación del \\ sistema seleccionado.\end{tabular}                                        & 15             & 1.4          \\ \cline{2-4} 
  \multicolumn{1}{|l|}{}                                                                                                                                                    & \begin{tabular}[c]{@{}l@{}}2.2 Evaluación de los \\ parametros seleccionados\\  en el sistema.\end{tabular}                    & 10             & 2.1          \\ \hline
  \multicolumn{1}{|l|}{\multirow{2}{*}{\begin{tabular}[c]{@{}l@{}}3. Evaluar la plataforma \\ seleccionada\end{tabular}}}                                                   & \begin{tabular}[c]{@{}l@{}}3.1 Evaluación de las \\ configuraciones del \\ sistema.\end{tabular}                               & 15             & 2.2          \\ \cline{2-4} 
  \multicolumn{1}{|l|}{}                                                                                                                                                    & \begin{tabular}[c]{@{}l@{}}3.2 Evaluación del \\ desempeño del sistema \\ mediante las métricas \\ seleccionadas.\end{tabular} & 12             & 3.1          \\ \hline
  \end{tabular}
  \end{table}


\section{Cronograma de actividades}

La agenda de actividades según el tiempo asignado comprende desde el  5 de Febrero del 2024 hasta el 25 de Mayo del 2024, tiempo el cual abarca 16 semanas lectivas en las cuales se llevara a cabo el desarrollo del proyecto. Por un lado el cronograma se presenta como un diagrama de Gantt, el mismo se puede observar en la Figura \ref{fig:gantt}. Por otro lado en la Figura \ref{fig:pert} se muestra el diagrama de PERT para lograr observar de una forma mas sencilla la dependencia entre actividades.
\newpage

\begin{figure}
  \centering
  \includegraphics[scale=0.3, angle=90]{diagramas/gantt.png}
  \caption{Ruta critica del cronograma de actividades}
  \label{fig:gantt}
\end{figure}

\begin{figure}
  \centering
  \includegraphics[scale=0.3, angle=90]{diagramas/pert.png}
  \caption{Diagrama PERT de las actividades}
  \label{fig:pert}
\end{figure}


\subsection{Entregables}

Como se puede observar en la Tabla \ref{tab:entregables} se encuentra una lista de los entregables con sus objetivos y fechas de entrega establecidas en el periodo donde se llevara a cabo el trabajo final de graduación.

\begin{table}[ht]
  \centering
  \caption{Lista de entregables}
  \label{tab:entregables}
  \begin{tabular}{|l|l|l|}
  \hline
  Objetivo         & Entregable                                                                                                                                                                & Fecha de entrega \\ \hline
  Específico 1     & \begin{tabular}[c]{@{}l@{}}Matriz de pugh con al menos \\ 6 criterios a evaluar del sistema \\ en tiempo real y el sistema \\ seleccionado.\end{tabular}                  &                  \\ \hline
  Específico 2     & El sistema implementado                                                                                                                                                   &                  \\ \hline
  Específico 3     & \begin{tabular}[c]{@{}l@{}}Métricas de calidad en donde \\ se define cual confuiguración \\ es mejor según la métrica \\ seleccionada.\end{tabular}                       &                  \\ \hline
  Objetivo general & \begin{tabular}[c]{@{}l@{}}Plataforma de trabajo en tiempo\\ real para un sistema embebido\\ que logre ejecutar un sistema de \\ control automático neuronal\end{tabular} &                  \\ \hline
  \end{tabular}
  \end{table}


\section{Uso de recursos}

Para ejecutar el software encargado de controlar la planta prototipo por medio de aprendizaje reforzado se hace uso de la siguiente lista de materiales.

\begin{itemize}
    \item Una planta prototipo de control automático donde se implemente el controlador.
    \item Computadora donde desarrollar el controlador. En caso de ser portátil, se requiere accesorios como el cargador.
    \item Tarjeta de desarrollo NVIDIA Jetson TX2.
    \item Acceso a internet para llevar a cabo las revisión bibliográfica.
    \item Tiempo de cómputo para el entrenamiento de las redes neuronales
\end{itemize}

\newpage

\section{Presupuesto}

\begin{figure}[ht]
  \centering
  \includegraphics[scale=0.8]{tablas/ptot.png}
  \captionsetup{labelformat=empty}  % Eliminar la etiqueta predeterminada ("Figura")
  \caption{Tabla 4: Presupuesto para realizar el proyecto}
\end{figure}

\begin{figure}[ht]
  \centering
  \includegraphics[scale=0.8]{tablas/pmensual.png}
  \captionsetup{labelformat=empty}  % Eliminar la etiqueta predeterminada ("Figura")
  \caption{Tabla 5: Presupuesto Mensual para realizar el proyecto}
\end{figure}


\newpage

\bibliography{biblio.bib}
\bibliographystyle{plain}

\end{document}