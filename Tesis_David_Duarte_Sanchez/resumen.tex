\chapter*{Resumen}
\thispagestyle{empty}

El resumen es la síntesis de lo que aparece en el resto del
documento. Tiene que ser lo suficientemente conciso y claro para que
alguien que lo lea sepa qué esperar del resto del trabajo, y se motive
para leerla completamente.  Usualmente resume lo más relevante de la
introducción y contiene la conclusión más importante del trabajo.

Es usual agregar palabras clave, que son los temas principales
tratados en el documento. El resumen queda fuera de la numeración del
resto de secciones.

Evite utilizar referencias bibliográficas, \tablas, o figuras en el
resumen.

\bigskip

%% Defina las palabras clave con defKeywords en config.tex:
\textbf{Palabras clave:} \thesisKeywords

\clearpage
\chapter*{Abstract}
\thispagestyle{empty}

Same content as the Spanish version, just in English.  Check
\href{https://deepl.com}{this site} for some help with the
translation.  For instance, the following is the automatic translation
from a previous version of the ``Resumen''.

The abstract is the synthesis of what appears in the rest of the
document. It has to be concise and clear enough so that someone
reading it knows what to expect from the rest of the text, and is
motivated to read it in full.  It usually summarizes the most relevant
parts of the introduction and contains the most important conclusion of
the work.

It is usual to add keywords, which are the main topics covered in the
document. The abstract is left out of the numbering of the rest of the
sections.

Avoid using bibliographical references, tables, or figures in the
abstract.

\bigskip

\textbf{Keywords:} word 1, word 2, 

\cleardoublepage

%%% Local Variables: 
%%% mode: latex
%%% TeX-master: "main"
%%% End: 
