\chapter*{Resumen}
\thispagestyle{empty}

Este proyecto se centra en el desarrollo de flujos de trabajo robustos para la implementación de software en sistemas de computadoras de guía, navegación y control (GNC) espaciales. El primer paso consiste en seleccionar una plataforma de hardware óptima para diseñar un modelo de ingeniería de una computadora de navegación espacial. Sobre esta base, se establecerán procesos para el prototipado de algoritmos de control de orientación y navegación, con un enfoque especial en sistemas hardware-in-the-loop. Estos sistemas permitirán una validación más realista de los algoritmos en condiciones espaciales simuladas.
Además, se explorarán casos de uso aplicados a un sistema de navegación y control a través del desarrollo de una aplicación de referencia. Esta aplicación se basará en una Unidad de Medida Inercial (IMU) y un controlador PID, demostrando la interacción y el rendimiento de los componentes involucrados. Este enfoque integral busca optimizar la implementación de software en sistemas GNC espaciales, mejorando significativamente su eficiencia y confiabilidad.

\bigskip

%% Defina las palabras clave con defKeywords en config.tex:
\textbf{Palabras clave:} \thesisKeywords

\clearpage
\chapter*{Abstract}
\thispagestyle{empty}

This project focuses on developing robust workflows for the implementation of software in space guidance, navigation, and control (GNC) computers. The first step involves selecting an optimal hardware platform to design an engineering model of a space navigation computer. Based on this foundation, processes will be established for prototyping orientation and navigation control algorithms, with a special emphasis on hardware-in-the-loop systems. These systems will enable a more realistic validation of the algorithms under simulated space conditions.
Additionally, use cases applied to a navigation and control system will be explored through the development of a reference application. This application will be based on an Inertial Measurement Unit (IMU) and a PID controller, demonstrating the interaction and performance of the involved components. This comprehensive approach aims to optimize the implementation of software in space GNC systems, significantly improving their efficiency and reliability.

\bigskip

\textbf{Keywords:} GNC, Systems, embedded, processor, framework, model to model transformation, embedded code, MATLAB, simulink, containers

\cleardoublepage

%%% Local Variables: 
%%% mode: latex
%%% TeX-master: "main"
%%% End: 
