\chapter{Conclusiones}

El proyecto de desarrollar flujos de trabajo para la implementación de software a bordo de computadoras de guía, navegación y control espacial ha logrado avances significativos en varios aspectos clave. Primero, la identificación de una plataforma de hardware adecuada para el desarrollo de un modelo de ingeniería de una computadora de navegación espacial ha sido un paso crucial. Esta plataforma no solo ha proporcionado un marco sólido para el diseño y la prueba, sino que también ha permitido optimizar los recursos y reducir los tiempos de desarrollo, lo que a su vez ha aumentado la eficiencia general del proyecto.

El establecimiento de flujos de trabajo para el prototipado de algoritmos de control de orientación y navegación con hardware en el loop ha sido otro logro notable. Estos flujos de trabajo han permitido una integración más fluida entre el software y el hardware, lo que ha mejorado significativamente la precisión y la estabilidad de los sistemas de navegación espacial. Además, la capacidad de probar y ajustar estos algoritmos en tiempo real ha acelerado el proceso de desarrollo y ha reducido el riesgo de errores críticos.

La evaluación de los casos de uso de una computadora de navegación y control espacial a través de la implementación de una aplicación de referencia demostrativa, específicamente el caso de la Unidad de Medida Inercial (IMU), ha proporcionado valiosos insights prácticos. Esta aplicación ha demostrado la viabilidad y el rendimiento de los flujos de trabajo desarrollados, validando la efectividad de los algoritmos y la plataforma de hardware seleccionada. Este enfoque práctico ha permitido identificar y abordar desafíos reales, asegurando que el sistema final sea robusto y confiable.

En resumen, el proyecto ha alcanzado sus objetivos específicos de manera satisfactoria, lo que ha sentado las bases para futuras innovaciones en el campo de la navegación y control espacial. La combinación de una plataforma de hardware adecuada, flujos de trabajo eficientes para el prototipado y la evaluación práctica de los casos de uso ha asegurado que el proyecto esté bien posicionado para enfrentar los desafíos complejos de la exploración espacial. Estos logros no solo mejoran las capacidades actuales sino que también abren caminos para investigaciones y aplicaciones futuras en este campo.
