%%%%%%%%%%%%%%%%%%%%%%%%%%%%%%%%%%%%%%%%%%%%%%%%%%%%%%%%%%%%%%%%%%%%%%%%%%%%%%%
% Author:  David Duarte
%
% Escuela de Ingeniería Electrónica
% Instituto Tecnológico de Costa Rica
%
% Tesis de Licenciatura
% 
% Phone:   +506 60348656
% email:   davidduarte28@gmail.com
%
%%%%%%%%%%%%%%%%%%%%%%%%%%%%%%%%%%%%%%%%%%%%%%%%%%%%%%%%%%%%%%%%%%%%%%%%%%%%%%%

% \documentclass is book
% If you want a printable two-side version of the thesis
%\documentclass[12pt,twoside,letterpaper]{book}

% If you want an electronic-only version of the thesis, do it one-sided
\documentclass[12pt,oneside,letterpaper]{book}

\usepackage[T1]{fontenc}
%\usepackage[utf8]{inputenc}   % is no longer required (since 2018)
\usepackage{ifthen}            % provide if-then-else operators

% --------------------------------------------------------------------------
% Global variables required in document formatting
% --------------------------------------------------------------------------
%
% BOOK MODE
%
\newboolean{bookmode}                  % boolean used to control book format
% Ensure that only one of the next two lines is active:
\setboolean{bookmode}{true}           % turn book mode on
%\setboolean{bookmode}{false}           % turn book mode off

%
% DRAFT MODE
%   The draft mode activates the TODO index and some "draft" markings all
%   around.  Ensure you set it to false for the final version!!
%   
%
\newboolean{draftmode}                  % boolean used to control draft-mode

%% -------------------------------------------------------------------------

%% Configure su nombre, título de tesis, lectores, fechas, etc. en:
%%
%% >>>>  config.tex  <<<<
%%

% --------------------------------------------------------------------------

% include all packages and define all required general macros
%\usepackage{fontspec}
\usepackage[spanish]{babel}
\usepackage[utf8]{inputenc}
\usepackage[scaled]{helvet}
\renewcommand\familydefault{\sfdefault} 
\usepackage[T1]{fontenc}
\usepackage{textcomp}
\usepackage{multicol}
\usepackage{array}                      % extensions to tabular environment
\usepackage{tabularx}                   % supports tables with fixed width
\usepackage{booktabs}
\usepackage{rotating}
\usepackage{trsym} %% Para simbolos de transformadas o---o
\usepackage{ifthen}
\usepackage{listings}
\usepackage{algorithm2e}
\usepackage{algorithmic}
\usepackage{icomma}
\usepackage{mathtools}
\usepackage{amsmath}
\usepackage{amssymb}
\usepackage{amstext}
\usepackage{bm}   %% Bold math

% Configuration of listings
\lstset{%
  basicstyle=\scriptsize,
  commentstyle=\color{blue}
}
\lstset{literate=%
  {á}{{\'a}}1
  {é}{{\'e}}1
  {í}{{\'i}}1
  {ó}{{\'o}}1
  {ú}{{\'u}}1
  {ñ}{{\~n}}1
  {Á}{{\'A}}1
  {É}{{\'E}}1
  {Í}{{\'I}}1
  {Ó}{{\'O}}1
  {Ú}{{\'U}}1
  {Ñ}{{\~N}}1
}

\definecolor{links}{HTML}{2A1B81}
\hypersetup{colorlinks,linkcolor=,urlcolor=links}

\DeclareMathAlphabet{\mathpzc}{OT1}{pzc}{m}{it}
\DeclareMathAlphabet{\mathpss}{OT1}{cmss}{m}{sl}

\usetheme[width=25mm]{Marburg}   % Barra azul a la derecha
\usecolortheme{tec}

\addtobeamertemplate{navigation symbols}{}{%
    \usebeamerfont{footline}%
    \usebeamercolor[fg]{footline}%
    \hspace{1em}%
    \raisebox{1pt}{\insertframenumber/\inserttotalframenumber}
}

\definecolor{tecAzul}{RGB}{31,47,95}      % según manual de imagen 2016
\definecolor{tecRojo}{RGB}{239,64,52}     % según manual de imagen 2016
\definecolor{tecNaranja}{RGB}{255,174,57} % según manual de imagen 2016
\definecolor{tecVerde}{RGB}{133,189,64}   % según manual de imagen 2016
\definecolor{tecCian}{RGB}{37,158,205}    % según manual de imagen 2016
\definecolor{tecTurq}{RGB}{35,168,156}    % según manual de imagen 2016

\definecolor{dkred}{RGB}{239,64,52}       % dark red
\definecolor{dkgreen}{RGB}{133,189,64}    % dark green
\definecolor{dkblue}{RGB}{31,47,95}       % dark blue
\definecolor{dkgray}{gray}{0.4}           % dark gray
\definecolor{ltgray}{gray}{0.75}          % light gray
\definecolor{dkmagenta}{rgb}{0.3,0.0,0.3} % dark magenta
\definecolor{ltyellow}{RGB}{251,75,58}    % light yellow
\definecolor{dkyellow}{RGB}{255,174,57}   % dark yellow
\definecolor{dkcyan}{RGB}{37,158,205}     % dark cyan
\definecolor{ltcyan}{rgb}{0.7,1,1}        % light cyan

\newcommand*{\vcenteredhbox}[1]{\begingroup
\setbox0=\hbox{#1}\parbox{\wd0}{\box0}\endgroup}

\newcommand{\bG}[1]{\textcolor{dkgreen}{\textbf{#1}}}
\newcommand{\bR}[1]{\textcolor{dkred}{\textbf{#1}}}
\newcommand{\bB}[1]{\textcolor{dkblue}{\textbf{#1}}}
\newcommand{\bM}[1]{\textcolor{dkmagenta}{\textbf{#1}}}
\newcommand{\bY}[1]{\textcolor{dkyellow}{\textbf{#1}}}
\newcommand{\bC}[1]{\textcolor{dkcyan}{\textbf{#1}}}

\newcommand{\iG}[1]{\textcolor{dkgreen}{\emph{#1}}}
\newcommand{\iR}[1]{\textcolor{dkred}{\emph{#1}}}
\newcommand{\iB}[1]{\textcolor{dkblue}{\emph{#1}}}
\newcommand{\iM}[1]{\textcolor{dkmagenta}{\emph{#1}}}
\newcommand{\iY}[1]{\textcolor{dkyellow}{\emph{#1}}}
\newcommand{\iC}[1]{\textcolor{dkcyan}{\emph{#1}}}

\newcommand{\nG}[1]{\textcolor{dkgreen}{{#1}}}
\newcommand{\nlG}[1]{\textcolor{ltgray}{{#1}}}
\newcommand{\ndG}[1]{\textcolor{dkgray}{{#1}}}
\newcommand{\nR}[1]{\textcolor{dkred}{{#1}}}
\newcommand{\nB}[1]{\textcolor{dkblue}{{#1}}}
\newcommand{\nM}[1]{\textcolor{dkmagenta}{{#1}}}
\newcommand{\nY}[1]{\textcolor{dkyellow}{{#1}}}
\newcommand{\nC}[1]{\textcolor{dkcyan}{{#1}}}


%%%%%%%%%%%%%%%%%%%%%%%%%%%%%%%%%%%%%%%%%%%%%%%%%%%%%%%%%%%%%%%%%%%%%%%%%%%%
%  Notación

\usepackage{mathrsfs}                   % Calygraphic fonts for transforms

%%%%%%%%%%%%%%%%%%%%%%%%%%%%%%%%%%%%%%%%%%%%%%%%%%%%%%%%%%%%%%%%%%%%%%%%%%%%

\def\hilite<#1>{%
  \temporal<#1>{\color{tecAzul!20!white}}{\color{tecRojo}}{\color{black}}}

\newcommand{\centrar}[1]{%
\begin{center}
  #1
\end{center}%
}

\deftranslation[to=spanish]{Example}{Ejemplo}
\deftranslation[to=spanish]{Definition}{Definición}

\newenvironment{exampleframe}[1][nothing]{
  \begin{frame}[allowframebreaks]
    \ifthenelse{\equal{#1}{nothing}}{
      \frametitle{Ejemplo}
    }{
      \frametitle{Ejemplo: #1}
    }
}{
  \end{frame}
}

\newcommand{\cleft}[2][.]{%
  \begingroup\colorlet{savedleftcolor}{.}%
  \color{#1}\left#2\color{savedleftcolor}%
}
\newcommand{\cright}[2][.]{%
  \color{#1}\right#2\endgroup
}


% define all symbols used in the document
%% (C) 2021 Pablo Alvarado
%% Escuela de Ingeniería Electrónica
%% Plantila para presentaciones
%% notation.tex
%% ---------------------------------------------------------------------------

%%
% Command definitions for localized symbol format definition
%%
\renewcommand{\Re}{\operatorname{Re}}
\renewcommand{\Im}{\operatorname{Im}}
\newcommand{\notimplies}{%
  \mathrel{{\ooalign{\hidewidth$\not\phantom{=}$\hidewidth\cr$\implies$}}}}

\newcommand{\prt}[1]{\ensuremath{\mathcal{#1}}}         %% partitioning
\newcommand{\img}[1]{\ensuremath{\mathcal{#1}}}         %% image as a set
\newcommand{\reg}[1][R]{\ensuremath{\mathcal{#1}}}      %% region
\newcommand{\pred}[1]{\ensuremath{\mathrm{#1}}}         %% predicate
\newcommand{\operat}[2]{\mathcal{#1}\left\{#2\right\}}
\newcommand{\transf}[1]{\mathscr{#1}}
\newcommand{\fourier}[1]{\transf{F}\left\{#1\right\}}
\newcommand{\ifourier}[1]{\transf{F}^{-1}\left\{#1\right\}}
\newcommand{\laplace}[1]{\transf{L}\left\{#1\right\}}
\newcommand{\ulaplace}[1]{\transf{L}_u\left\{#1\right\}}
\newcommand{\blaplace}[1]{\transf{L}_b\left\{#1\right\}}
\newcommand{\ilaplace}[1]{\transf{L}^{-1}\left\{#1\right\}}
\newcommand{\ztrans}[1]{\transf{Z}\left\{#1\right\}}
\newcommand{\iztrans}[1]{\transf{Z}^{-1}\left\{#1\right\}}
\newcommand{\zutrans}[1]{\transf{Z}_u\left\{#1\right\}}
\newcommand{\exceq}{\ensuremath{\overset{!}{=}}}

\newcommand{\tr}[1]{\operatorname{tr}\!\left\{#1\right\}}
\newcommand{\rank}{\operatorname{rg}}
\newcommand{\trace}[1]{\operatorname{tr}{#1}}
\newcommand{\signum}{\operatorname{signum}}
\newcommand{\pos}{\operatorname{pos}}
\newcommand{\val}{\operatorname{val}}
\newcommand{\Var}{\operatorname{Var}}
\newcommand{\Cov}{\operatorname{Cov}}
\newcommand{\Curt}{\operatorname{Curt}} % Curtosis
\newcommand{\E}{\operatorname{E}}
\newcommand{\ind}[1]{1\!\left\{#1\right\}} % indicator function
\newcommand{\vct}[1]{\ensuremath{\underline{\bm{{#1}}}}}
\newcommand{\pix}[1]{\ensuremath{\bm{#1}}}
\newcommand{\mat}[1]{\ensuremath{\bm{{#1}}}}
\newcommand{\tsr}[1]{\ensuremath{\mathcal{#1}}}
\newcommand{\jacob}{\ensuremath{\bm{J}}} % Jacobian
\newcommand{\VC}{\operatorname{VC}} % Vapnik-Chervonenkis
\newcommand{\MI}{\operatorname{MI}} % Mutual information
\newcommand{\KL}{\operatorname{KL}} % Kullback-Leibler
\newcommand{\MAP}{\operatorname{MAP}} % Maximum a posteriori

\newcommand{\vctOmega}{\vct{\bm{\Omega}}}
\newcommand{\vctalpha}{\vct{\bm{\alpha}}}
\newcommand{\vctbeta}{\vct{\bm{\beta}}}
\newcommand{\vctgamma}{\vct{\bm{\gamma}}}
\newcommand{\vcteta}{\vct{\bm{\eta}}}
\newcommand{\vctmu}{\vct{\bm{\mu}}}
\newcommand{\vctomega}{\vct{\bm{\omega}}}
\newcommand{\vctphi}{\vct{\bm{\phi}}}
\newcommand{\vctpsi}{\vct{\bm{\psi}}}
\newcommand{\vctpi}{\vct{\bm{\pi}}}
\newcommand{\vcttheta}{\vct{\bm{\theta}}}
\newcommand{\vctvarphi}{\vct{\bm{\varphi}}}
\newcommand{\vctsigma}{\vct{\bm{\sigma}}}
\newcommand{\vctxi}{\vct{\bm{\xi}}}
\newcommand{\vctzeta}{\vct{\bm{\zeta}}}
\newcommand{\vctvarepsilon}{\vct{\bm{\varepsilon}}}

\newcommand{\raum}[1]{\ensuremath{\mathbb{#1}}}

\newcommand{\matGamma}{\mat{\bm{\Gamma}}}
\newcommand{\matLambda}{\mat{\bm{\Lambda}}}
\newcommand{\matPhi}{\mat{\bm{\Phi}}}
\newcommand{\matPi}{\mat{\bm{\Pi}}}
\newcommand{\matPsi}{\mat{\bm{\Psi}}}
\newcommand{\matSigma}{\mat{\bm{\Sigma}}}
\newcommand{\matTheta}{\mat{\bm{\Theta}}}

\newcommand{\row}[2]{\ensuremath{\underline{\bm{#1}}_{#2,:}}}
\newcommand{\col}[2]{\ensuremath{\underline{\bm{#1}}_{:,#2}}}
\newcommand{\seq}[1]{\ensuremath{#1}}
\newcommand{\set}[1]{\ensuremath{\mathcal{#1}}}
\newcommand{\gset}[1]{\ensuremath{#1}} %% set for greek symbols
\newcommand{\front}[1]{\widehat{\set{#1}}}
\newcommand{\setlambda}{\set{\bm{\lambda}}}
\newcommand{\klass}[1]{\ensuremath{\mathpss{#1}}}
\newcommand{\graph}[1]{\ensuremath{\mathsf{#1}}}
\newcommand{\lab}[1]{\ensuremath{\mathpss{L}(#1)}}
\newcommand{\myfrac}[2]{{\footnotesize #1/#2}}
\newcommand{\ifthenspc}{\rule{3mm}{0mm}}
\newcommand{\point}[1]{\ensuremath{\mathsf{#1}}}
\newcommand{\estim}[1]{\ensuremath{\hat{#1}}}
\newcommand{\numset}[1]{\ensuremath{\mathbb{#1}}}
\newcommand{\tuple}[1]{\ensuremath{\left\langle#1\right\rangle}}
\newcommand{\conj}[1]{\ensuremath{{{#1}^{\ast}}}}
\newcommand{\base}[1]{\set{#1}}
\newcommand{\zeron}[1]{\ensuremath{\underset{\uparrow}{#1}}}
\newcommand{\sysT}{\ensuremath{\mathcal{T}}}
\newcommand{\sys}[1]{\ensuremath{\sysT\left\{#1\right\}}}
%\newcommand{\sen}{\operatorname{sen}} % sinus in spanish (seno)
%\newcommand{\senh}{\operatorname{senh}} % sinus hiperbolicus in spanish (seno)
%\newcommand{\arcsen}{\operatorname{arcsen}} % arcus sinus hiperbolicus in spanish (arcoseno)
\newcommand{\sgn}{\operatorname{sgn}} % signus
\newcommand{\roc}{\text{ROC: }}
\newcommand{\med}{\operatorname{med}} % median
\newcommand{\diag}{\operatorname{diag}} % diagonal
\newcommand{\normal}{\ensuremath{\mathcal{N}}}
\newcommand{\binomial}{\ensuremath{\mathcal{B}}}
\newcommand{\multinomial}{\ensuremath{\operatorname{Multinomial}}}
\newcommand{\Bernoulli}{\operatorname{Ber}}
\newcommand{\drawn}{\thicksim}

\newcommand{\lagr}{\ensuremath{\mathcal{L}}}
\newcommand{\primal}{\ensuremath{\mathcal{P}}}
\newcommand{\original}{\ensuremath{\mathcal{P}}}
\newcommand{\dual}{\ensuremath{\mathcal{D}}}

\newcommand{\code}[1]{\texttt{#1}}
\newcommand{\conv}{\ensuremath{\ast}}
\newcommand{\corr}{\ensuremath{\circ}}
\newcommand{\cconv}{\ensuremath{\;\,\text{\footnotesize{N}}\!\!\!\!\!\!\bigcirc}}
\newcommand{\Ln}{\operatorname{Ln}}
\newcommand{\sa}{\operatorname{sa}}
\newcommand{\senc}{\operatorname{senc}}
\newcommand{\si}{\operatorname{si}}


%% Natural, Integer and Real Numbers
\newcommand{\setA}{\ensuremath{\mathbb{A}}}
\newcommand{\setB}{\ensuremath{\mathrm{I\negthinspace B}}}
\newcommand{\setC}{\ensuremath{\mathbb{C}}}
\newcommand{\setD}{\ensuremath{\mathrm{I\negthinspace D}}}
\newcommand{\setE}{\ensuremath{\mathrm{I\negthinspace E}}}
\newcommand{\setF}{\ensuremath{\mathrm{I\negthinspace F}}}
\newcommand{\setG}{\ensuremath{\mathbb{G}}}
\newcommand{\setH}{\ensuremath{\mathrm{I\negthinspace H}}}
\newcommand{\setI}{\ensuremath{\mathbb{I}}}
\newcommand{\setJ}{\ensuremath{\mathbb{J}}}
\newcommand{\setK}{\ensuremath{\mathrm{I\negthinspace K}}}
\newcommand{\setL}{\ensuremath{\mathrm{I\negthinspace L}}}
\newcommand{\setM}{\ensuremath{\mathrm{I\negthinspace M}}}
\newcommand{\setN}{\ensuremath{\mathrm{I\negthinspace N}}}
\newcommand{\setO}{\ensuremath{\mathbb{O}}}
\newcommand{\setP}{\ensuremath{\mathrm{I\negthinspace P}}}
\newcommand{\setQ}{\ensuremath{\mathbb{Q}}}
\newcommand{\setR}{\ensuremath{\mathrm{I\negthinspace R}}}
\newcommand{\setS}{\ensuremath{\mathbb{S}}}
\newcommand{\setT}{\ensuremath{\mathbb{T}}}
\newcommand{\setU}{\ensuremath{\mathbb{U}}}
\newcommand{\setV}{\ensuremath{\mathbb{V}}}
\newcommand{\setW}{\ensuremath{\mathbb{W}}}
\newcommand{\setX}{\ensuremath{\mathbb{X}}}
\newcommand{\setY}{\ensuremath{\mathbb{Y}}}
\newcommand{\setZ}{\ensuremath{\mathbb{Z}}}

\newcommand{\orden}{\ensuremath{\mathcal{O}}}
\newcommand{\order}{\ensuremath{\mathcal{O}}}
\newcommand{\loss}{\ensuremath{\mathcal{L}}}

%%
% Multimap symbols
%
\newcommand{\ttoF}{\TransformHoriz}
\newcommand{\Ftot}{\InversTransformHoriz}
\newcommand{\ttoZ}{\ttoF}
\newcommand{\Ztot}{\Ftot}
\newcommand{\ttoZu}{\overset{z_u}{\ttoF}}
\newcommand{\Zutot}{\overset{z_u}{\Ftot}}
\newcommand{\vttoF}{\text{\begin{sideways}$\Ftot$\end{sideways}}}
\newcommand{\vFtot}{\text{\begin{sideways}$\ttoF$\end{sideways}}}
\newcommand{\vttoZ}{\vttoF}
\newcommand{\vZtot}{\vFtot}
\newcommand{\ttoDF}{\underset{N}{\ttoF}}
\newcommand{\DFtot}{\underset{N}{\Ftot}}

\newcommand{\thisis}[2]{\underset{#1}{\underbrace{#2}}}

\newcommand\givenbase[1][]{\:#1\lvert\:}
\let\given\givenbase
\newcommand\sgiven{\givenbase[\delimsize]}
\DeclarePairedDelimiterX\Basics[1](){\let\given\sgiven #1}
                    

\newcommand{\alphaspanish}[1]{%
  \ifcase#1 a\or b\or c\or d\or e\or f\or g\or h\or i\or j\or k\or l\or m\or n\or o\or p\or q\or r\or s\or t\or u\or v\or w\or x\or y\or z\fi}


% allow equations to be splitted (breaked) into several pages
\allowdisplaybreaks[3]

% --------------------------------------------------------------------------
\begin{document}
  % where to look for graphics
  \graphicspath{{./}{./fig/}}

  \pagenumbering{alph}
  % fix some terms not activated due to the bug of hyperref with spanish.
  
  \renewcommand{\examplesolution}{Solución}
  \pagestyle{empty}

  % select one of the following titlepages
  \include{titlepage_licel_es} % Titlepage in Spanish
  %\include{titlepage_licel_en} % Titlepage in English (only if thesis is in En)

  %%% ---------------------------------------------------------------------------
%% titlepage_msc_es.tex
%%
%% Title page
%%
%% $Id: titlepage.tex 1452 2010-07-07 00:55:16Z palvarado $
%% ---------------------------------------------------------------------------
\phantomsection
\pdfbookmark[1]{Portada}{Portada}

\thispagestyle{empty} 

\begin{center}

Tecnológico de Costa Rica

\par\vspace{1ex}

Escuela de Ingeniería Electrónica

\par\vspace{20mm}

\includegraphics[height=60mm]{Firma_TEC-4}

\par\vspace*{\fill}

{\large\bf{\thesisTitle\par}}

\par\vspace*{\fill}

Documento de tesis sometido a consideración para optar por el grado
académico de Maestría en Electrónica con Énfasis en
%
%Sistemas Embebidos
Procesamiento Digital de Señales
%Microelectrónica
%Sistemas Microelectromecánicos

\par\vspace{20mm}

\thesisAuthor

\vspace*{\fill}

\ifdraft{%
  {Borrador de \thesisDraftDate}%
}{%
  {Cartago, \thesisFinalDate}%
}
\end{center}
\newpage 
\cleardoublepage 


%%% Local Variables: 
%%% mode: latex
%%% TeX-master: "main"
%%% End: 
   % Titlepage in Spanish
  %\include{titlepage_msc_en}   % Titlepage in English (only if thesis is in En)

  \thispagestyle{empty}

\rule{\textwidth}{0pt}

\vfill

\ifdraft{%
  El documento
  \href{https://www.tec.ac.cr/sites/default/files/media/doc/requisitos_trabajos_finales_graduacion_2021.pdf}%
  {Requisitos para la entrega de Trabajos Finales de Graduación} a las
  bibliotecas del TEC indica que usted debe incluir la licencia de
  Creative Commons en la página siguiente de la portada.

  Asegúrse entonces de \href{https://creativecommons.org/choose/?lang=es}%
  {elegir la licencia correcta}, y ajustar el texto abajo a su selección.

  Es necesario que
  \href{https://creativecommons.org/about/downloads/}{descargue el
    ícono} correcto en formato vectorial, y lo coloque en el
  directorio \code{fig/}.%
}



\vfill


\framebox[\textwidth]{
  \footnotesize
  \parbox{0.98\textwidth}{%
    \begin{center} %
      \includegraphics[scale=1]{by-sa} %
    \end{center} %
    
    Este trabajo titulado \emph{\thesisFlatTitle{}} por \thesisAuthor{}, se
    encuentra bajo la Licencia Creative Commons
    \href{http://creativecommons.org/licenses/by-sa/4.0/?ref=chooser-v1}%
    {Atribución-ShareAlike 4.0 International}.
    
    Para ver una copia de esta Licencia, visite
    \url{http://creativecommons.org/licenses/by-sa/4.0/}.\bigskip
    
    \copyright \the\year \hfill%
    \thesisAuthor \hfill%
    \thesisInstitution
  }
}

  
  % Hoja de depuración, con comandos definidos por la plantilla.
  % \phantomsection
\pdfbookmark[1]{Debug}{Debug}

Esta es una página de depuración, para ver todos los comandos
definidos en config.tex

De \verb+babel+ se obtiene que \verb+\tablename+ es \tablename.  Por
lo tanto, en esta versión se usará ``\latabla'' para denotar a cada
``\tabla''.  Ver \tabref{tab:comandostab} para la lista de comandos
existentes.



Este documento es elaborado por \thesisAuthorAddress~\thesisAuthor\
(\thesisAuthorShort) con carné \thesisAuthorTECID, para optar por el
título de \thesisAuthorDegree.

\genderAsesor\ \nameAsesor.

\genderLectorI\ \nameLectorI.

\genderLectorII\ \nameLectorII.

Titulo crudo:

\begin{center}
  \thesisTitle.  
\end{center}

Título aplanado:

``\thesisFlatTitle''.

Palabras clave: \thesisKeywords.

Fecha borrador: \thesisDraftDate

Fecha final: \thesisFinalDate





  
  \include{disclaimer}
  %% -------------------------------------------------
  %% Acta y hoja del tribunal
  %%
  %% Asegúrse de que las fechas de defensa de tesis sean las que aparecen
  %% en las actas.
  
  %% Para la Licenciatura en Ingeniería Electrónica:

  %%   Acá se colocan las dos actas como plantillas para ser firmadas
  %%   por el tribunal.

  %%   El acta de aprobación, dependiendo del tribunal, puede dejarla
  %%   en blanco en la tesis, para que el tribunal firme la tesis completa
  %%   sobre esta acta, o, si el tribunal lo decide, extrae la hoja
  %%   para que sea firmada "caligráficamente" por los miembros del tribunal.
  %%   Al acta firmada, en formato PDF (ya sea firmado con tabletas gráficas o
  %%   en papel y escaneada) la integra al documento con el comando para incluir
  %%   el pdf directamente \includepdf{archivo} 
  %% ESTE ARCHIVO DEBE ELIMINARSE DE LA VERSIÓN FINAL

\thispagestyle{empty}

\begin{center}
  \begin{tabular}{c}
    \thesisInstitution \\
    \thesisDepartment \\
    Trabajo Final de Graduación \\
    Acta de Aprobación
  \end{tabular}
\end{center}

\vfill

\begin{center}
  \begin{tabular}{c}
    Defensa de Trabajo Final de Graduación \\
    Requisito para optar por el título de \thesisAuthorDegree\ en Electrónica\\
    Grado Académico de Licenciatura
  \end{tabular}
\end{center}

\vfill

%% \thesisAuthorAddress, \thesisAuthor y \thesisTitle están en main.tex
El Tribunal Evaluador aprueba la defensa del trabajo final de graduación
denominado \textsl{\thesisFlatTitle{}}, realizado por
%
\thesisAuthorAddress\ \thesisAuthor\ %
%
y, hace constar que cumple con las normas
establecidas por la \thesisDepartment{} del \thesisInstitution{}.

\vfill

\begin{center}
 Miembros del Tribunal Evaluador
\end{center}

\vfill

\begin{center}
  \begin{tabularx}{\textwidth}{cXc}
    \rule{0.45\textwidth}{0.5pt} && \rule{0.45\textwidth}{0.5pt} \\
    \nameLectorI                 && \nameLectorII \\
    \genderLectorI               && \genderLectorII
  \end{tabularx}
  
  \vspace{10mm}

  \begin{tabular}{c}
    \rule{0.45\textwidth}{0.5pt} \\
    \nameAsesor \\
    \genderAsesor
  \end{tabular}
\end{center}

\vfill

\begin{center}
  Cartago, \ifdraft{\thesisDraftDate}{\thesisFinalDate}\par
\end{center}

\cleardoublepage

%%% Local Variables: 
%%% mode: latex
%%% TeX-master: "main"
%%% End: 
  % Remover en versión final
  %\includepdf{acta_aprob_firmada} % Incluir el acta firmada acá.

  %% El acta de evaluación usualmente la extrae del documento y
  %% la entrega al tribunal para que sea firmada, y ellos la hacen
  %% llegar al profesor del curso de TFG.  Ese documento no
  %% debe aparecer en la tesis final, así que esta línea deberá
  %% comentarla en la versión final:
  %% ESTE ARCHIVO DEBE ELIMINARSE DE LA VERSIÓN FINAL


\thispagestyle{empty}

\begin{center}
  \begin{tabular}{c}
    \thesisInstitution \\
    \thesisDepartment \\
    Trabajo Final de Graduación \\
    Tribunal Evaluador \\
    Acta de Evaluación
  \end{tabular}
\end{center}

\vfill

\begin{center}
  \begin{tabular}{c}
    Defensa del Trabajo Final de Graduación \\
    Requisito para optar por el título de \thesisAuthorDegree\ en Electrónica\\
    Grado Académico de Licenciatura
  \end{tabular}
\end{center}

\vfill

%% Configurar todo en config.tex
\begin{center}

  Estudiante:%
  \qquad \textbf{\thesisAuthor}%
  \qquad Carné: \thesisAuthorTECID

  \vspace*{2ex}

  \setlength\tabcolsep{0pt}
  \begin{tabular}{p{.25\textwidth}p{.73\textwidth}}
    Nombre del proyecto: & \textsl{\thesisFlatTitle}
  \end{tabular}
\end{center}
\vspace{5mm}

\vfill

Los miembros de este Tribunal hacen constar que este trabajo final de
graduación ha sido aprobado y cumple con las normas establecidas por
la \thesisDepartment{} del \thesisInstitution{} y es merecedor de la
siguiente calificación:

\vfill

\begin{center}
  Nota del Trabajo Final de Graduación: \rule{25mm}{0.5pt}
\end{center}

\vfill

\begin{center}
 Miembros del Tribunal Evaluador
\end{center}

\vfill

% Defina con \setLector* en main.tex (líneas 87-89) los lectores y asesor
\begin{center}
  \begin{tabularx}{\textwidth}{cXc}
    \rule{0.45\textwidth}{0.5pt} && \rule{0.45\textwidth}{0.5pt} \\
    \nameLectorI                 && \nameLectorII \\
    \genderLectorI               && \genderLectorII
  \end{tabularx}
  
  \vspace{10mm}

  \begin{tabular}{c}
    \rule{0.45\textwidth}{0.5pt} \\
    \nameAsesor \\
    \genderAsesor
  \end{tabular}
\end{center}

\vfill

\begin{center}
  Cartago, \ifdraft{\thesisDraftDate}{\thesisFinalDate}\par
\end{center}

\cleardoublepage

%%% Local Variables: 
%%% mode: latex
%%% TeX-master: "main"
%%% End: 
   % >> Remover en versión final <<
    
  %% Para la maestría en electrónica:
  %\include{acta_aprob_mscel}  % Remover en versión final
  %%% ESTE ARCHIVO DEBE ELIMINARSE DE LA VERSIÓN FINAL


\thispagestyle{empty}

\begin{center}
  \begin{tabular}{c}
    \thesisInstitution \\
    \thesisDepartment \\
    Tesis de Maestría \\
    Acta de Evaluación
  \end{tabular}
\end{center}

\vfill

Tesis de maestría defendida ante el presente Tribunal Evaluador como
requisito para optar por el grado académico de maestría, del
\thesisInstitution.

\vspace*{15mm}

%% Configurar todo en config.tex
\begin{center}
  Estudiante: \thesisAuthor
\end{center}

\vfill

\begin{center}
  Nombre del Proyecto: \thesisFlatTitle}
\end{center}

\vspace*{20mm}
\begin{center}
 Miembros del Tribunal Evaluador
\end{center}
\vspace*{8mm}

\vfill

% Los nombres de lectores y asesor se definen en el archivo main.tex
\begin{center}
  \begin{tabularx}{\textwidth}{cXc}
    \rule{0.45\textwidth}{0.5pt} && \rule{0.45\textwidth}{0.5pt} \\
    \nameLectorI                 && \nameLectorII \\
    \genderLectorI               && \genderLectorII
  \end{tabularx}
  
  \vspace{10mm}

  \begin{tabular}{c}
    \rule{0.45\textwidth}{0.5pt} \\
    \nameAsesor \\
    \genderAsesor
  \end{tabular}
\end{center}

\vfill

Los miembros de este Tribunal dan fe de que la presente tesis de
maestría ha sido aprobada y cumple con las normas establecidas por la
\thesisDepartment.

\vfill

\begin{center}
  Nota final de la Tesis de Maestría: \rule{3cm}{0.5pt}
\end{center}
\vfill

\begin{center}
  Cartago, \today\par
\end{center}

\cleardoublepage

%%% Local Variables: 
%%% mode: latex
%%% TeX-master: "main"
%%% End: 
   % Remover en versión final
  %% -------------------------------------------------
  \chapter*{Resumen}
\thispagestyle{empty}

Este proyecto se centra en el desarrollo de flujos de trabajo robustos para la implementación de software en sistemas de computadoras de guía, navegación y control (GNC) espaciales. El primer paso consiste en seleccionar una plataforma de hardware óptima para diseñar un modelo de ingeniería de una computadora de navegación espacial. Sobre esta base, se establecerán procesos para el prototipado de algoritmos de control de orientación y navegación, con un enfoque especial en sistemas hardware-in-the-loop. Estos sistemas permitirán una validación más realista de los algoritmos en condiciones espaciales simuladas.
Además, se explorarán casos de uso aplicados a un sistema de navegación y control a través del desarrollo de una aplicación de referencia. Esta aplicación se basará en una Unidad de Medida Inercial (IMU) y un controlador PID, demostrando la interacción y el rendimiento de los componentes involucrados. Este enfoque integral busca optimizar la implementación de software en sistemas GNC espaciales, mejorando significativamente su eficiencia y confiabilidad.

\bigskip

%% Defina las palabras clave con defKeywords en config.tex:
\textbf{Palabras clave:} \thesisKeywords

\clearpage
\chapter*{Abstract}
\thispagestyle{empty}

This project focuses on developing robust workflows for the implementation of software in space guidance, navigation, and control (GNC) computers. The first step involves selecting an optimal hardware platform to design an engineering model of a space navigation computer. Based on this foundation, processes will be established for prototyping orientation and navigation control algorithms, with a special emphasis on hardware-in-the-loop systems. These systems will enable a more realistic validation of the algorithms under simulated space conditions.
Additionally, use cases applied to a navigation and control system will be explored through the development of a reference application. This application will be based on an Inertial Measurement Unit (IMU) and a PID controller, demonstrating the interaction and performance of the involved components. This comprehensive approach aims to optimize the implementation of software in space GNC systems, significantly improving their efficiency and reliability.

\bigskip

\textbf{Keywords:} GNC, Systems, embedded, processor, framework, model to model transformation, embedded code, MATLAB, simulink, containers

\cleardoublepage

%%% Local Variables: 
%%% mode: latex
%%% TeX-master: "main"
%%% End: 

  \vspace*{0.4\textheight}
% No debe confundirse la dedicatoria con el agradecimiento.
% La dedicatoria solo tiene una línea corta de la persona a quien se dedica.

{\hfill{\Large{\emph{a mis queridos padres}}}}

  \chapter*{Agradecimientos}
\thispagestyle{empty}

El resultado de este trabajo no hubiese sido posible sin el apoyo de Thevenin,
Norton, Einstein y mi querido amigo Ohm.

Usualmente se agradece aquí a la empresa o investigador que dio la
oportunidad de realizar el trabajo final de graduación.

No debe confundir el agradecimiento con la dedicatoria.  La
dedicatoria es usualmente una sola línea, con la persona a quien se
dedica el trabajo.

El agradecimiento es un texto más elaborado, de caracter personal, en
donde se expresa la gratitud por la oportunidad, el apoyo brindado, la
inspiración ofrecida, el acompañamiento moral, etc.

\vspace*{1cm}

\thesisAuthor

Cartago, \today

\cleardoublepage

%%% Local Variables: 
%%% mode: latex
%%% TeX-master: "paMain"
%%% End: 


  %----------------------------------------------------------------------------
  \frontmatter
  %----------------------------------------------------------------------------
  \pagestyle{fancy}
  \pagenumbering{roman}

  \pdfbookmark[1]{Indice General}{Indice General}

  \parskip0ex                           % space between paragraphs

  \tableofcontents                      % Table of contents
  \listoffigures                        % List of figures
  \listoftables                         % List of tables

\ifdraft{%
  % todo's                              % TODOs
  \listoftodo
}{%
}

  %% ---------------------------------------------------------------------------
%% paNotation.tex
%%
%% Notation
%%
%% $Id: paNotation.tex,v 1.15 2004/03/30 05:55:59 alvarado Exp $
%% ---------------------------------------------------------------------------

\cleardoublepage
\renewcommand{\nomname}{Lista de símbolos y abreviaciones}
\setlength{\nomitemsep}{-\parsep}

%%
% Commands required for the nomenclature groups
%
% There are following prefix forms:
%  a   abbreviation    \syma[key]{symbol}{description}
%  g   general         \symg[key]{symbol}{description}
%%

\renewcommand{\nomgroup}[1]{%
  \ifthenelse{\equal{#1}{G}}{\section*{\hspace*{-\leftmargin}Notación general}}{}%
  \ifthenelse{\equal{#1}{A}}{\section*{\hspace*{-\leftmargin}Abreviaturas y Siglas}}{}%
}

\newcommand{\syma}[3][foo]{%
  \ifthenelse{\equal{#1}{foo}}%
  {\nomenclature[A]{#2}{#3}}{\nomenclature[A#1]{#2}{#3}}}
\newcommand{\symg}[3][foo]{%
  \ifthenelse{\equal{#1}{foo}}%
  {\nomenclature[G]{#2}{#3}}{\nomenclature[G#1]{#2}{#3}}}

%%
% Símbolos en la notación general
% (es posible poner la declaración en el texto
%%


\symg[t]{$\sys{\cdot}$}{Transformación realizada por un sistema.}
\symg[yscalar]{$y$}{Escalar.}
\symg[zconjugado]{$\conj{z}$}{Complejo conjugado de $z$.}
\symg[rcomplexreal]{$\Re(z)$ o $z_{\Re}$}{Parte real del número complejo $z$.}
\symg[icompleximag]{$\Im(z)$ o $z_{\Im}$}{Parte imaginaria del número
                                        complejo $z$.}
\symg[jimaginario]{$j$}{$j=\sqrt{-1}$.}
\symg[xvector]{$\vct{x}$}{Vector. \newline\hspace{1mm}%
  $\vct{x}=\left[ x_1 \; x_2 \; \ldots \; x_n \right]^T =
  \begin{bmatrix}
    x_1 \\ x_2 \\ \vdots \\ x_n
  \end{bmatrix}$}

\symg[mmatrix]{$\mat{A}$}{Matriz. \newline\hspace{1mm}%
  $\mat{A} =
  \begin{bmatrix}
    a_{11} & a_{12} & \cdots & a_{1m}\\
    a_{21} & a_{22} & \cdots & a_{2m}\\
    \vdots & \vdots & \ddots & \vdots\\
    a_{n1} & a_{n2} & \cdots & a_{nm}\\
  \end{bmatrix}$}

\symg[C]{$\setC$}{Conjunto de los números complejos.}

%%
% Algunas abreviaciones
%%

\syma{PCA}{Análisis de componentes principales}
\syma{WSN}{Redes Inalámbricas de Sensores}
\syma{ASM}{Modelos Activos de Forma}

\printnomenclature[20mm]

%%% Local Variables:
%%% mode: latex
%%% TeX-master: "paMain"
%%% End:
                    % Abbreviation

  \parskip1.3ex                         % space between paragraphs

  %----------------------------------------------------------------------------
  \mainmatter
  %----------------------------------------------------------------------------
  % where to look for graphics
  \graphicspath{{./}{./fig/}}
  %\pagenumbering{arab}

  % Main files
  %% ---------------------------------------------------------------------------
%% intro.tex
%%
%% Introduction
%%
%% $Id: intro.tex 1477 2010-07-28 21:34:43Z palvarado $
%% ---------------------------------------------------------------------------

\chapter{Introducción}
\label{chp:intro}

En la \nt{introducción} deben quedar completamente claros los siguientes
aspectos, cuyo significado depende del tipo concreto de tesis:

\begin{compactitem}
\item Contexto
\item Antecedentes
\item Problema concreto
\item Esbozo de solución
\item Objetivos y estructura
\end{compactitem}

Una buena introducción debe lograr que el lector tenga interés de leer el resto
del tesis.

Es recomendable dividir la tesis en secciones, nombradas cada una de acuerdo a
su contenido. \textbf{Jamás} utilice los nombres de la guía como
``\emph{Problema existente e importancia de su solución}'', sino algo como ``La
deforestación en Costa Rica'' o lo que se adecúe a su problema en particular.

Recuerde que en español solo la primera letra del título va en mayúscula
(exceptuando nombres propios, por supuesto).
%
Algunos recursos adicionales a esta guía los encuentra en \cite{AlvaradoWeb}.


\section{El cambio climático y la electrónica}
\label{sec:contexto}

El contexto corresponde al entorno donde se desarrolla el proyecto de
tesis, que puede ser el área general de aplicación, un dominio de
problemas, etc.

De nuevo, no use un título genérico como ``Contexto'', sino algo
asociado directamente a su trabajo.


\section{Antecedentes}

Si su proyecto se circunscribe en otro proyecto mayor, en el que han
participado otros estudiantes de grado y postgrados, y ya existen
tesis o artículos publicados, en esta sección se hace una breve reseña
de esos trabajos previos, con el objetivo de contextualizar en dónde
calza concretamente el trabajo actual dentro de ese otro proyecto
mayor.  Por ejemplo, Fulano en~\cite{Fulano21} exploró si un diodo
puede funcionar como fuente de energía infinita, hipótesis que no
logró comprobar.

En proyectos relativamente aislados, no es necesaria esta sección.

Dependiendo de cada trabajo concreto, esta sección puede desplazarse a
otro lugar dentro de la introducción donde tenga más sentido, pero
usualmente se encuentra aquí justo antes de presentar el problema
técnico concreto tratado en su proyecto.

\section{La disipación de energía en el reactor 42}

En esta sección usted debe exponer su problema concreto.  Debe enlazar
el contexto general, expuesto en las secciones anteriores, con el
problema concreto que este trabajo resuelve.

Al final de esta sección, el problema concreto se sintetiza usualmente
en una frase de planteamiento del problema de ingeniería o pregunta
generadora de la investigación de ingeniería. Esta frase o pregunta
debería ser una consecuencia a la que se llega después de realizar el
desarrollo del contexto.  Si el problema es de caracter científico,
aquí puede plantearse la hipótesis de la investigación científica.

Del planteamiento del problema se deriva cuál es el objetivo del
trabajo en particular, que a su vez debe conducir al lector de forma
natural al esbozo de la solución del problema a tratar en este
informe.

\section{Sistema de almacenamiento energético}

Después de las secciones anteriores ya ha guiado al lector hasta este
punto en donde solo resta presentar una propuesta general de solución
del problema técnico concreto.

Para aclarar la solución se hace uso de un diagrama de bloques (ver
\figref{fig:diagbloques}) o diagrama de flujo general, es decir,
desde un nivel de abstracción muy alto, donde no sea necesario entrar
en detalles técnicos, porque aun no han sido expuestos.

\begin{figure}[htb]

  %% Este es un ejemplo de figura TIKZ incrustada directamente
  %% en el la figura, pero es muchísimo más recomendable poner este
  %% código, tal y como se explica en el siguiente capítulo, en un archivo
  %% aparte.  Como archivo aparte se puede compilar la figura una sola vez
  %% para que quede disponible en el directorio fig/.  Eso es más rápido de
  %% compilar posteriormente.
  %%
  %%
  %% Se coloca este ejemplo aquí porque muchas personas están usando
  %% Overleaf, y allí puede tener sentido tener las figuras
  %% directamente en el código del informe, aunque esto tardará más en
  %% compilar e incluso puede llevar a Overleaf a pasarse del tiempo
  %% disponible.
  
  \centering
  \tikzstyle{block} = [draw, rectangle, inner sep=6pt]
  \begin{tikzpicture}[>=latex,auto,node distance=2cm]
    \node [block](system) {Sistema};
    \node [coordinate, left=of system] (infork) {};
    \node [coordinate, left=of infork] (input) {};
    \node [coordinate, right=of system] (outfork) {};
    \node [coordinate, right=of outfork] (output) {};
    \node [block, below=of system] (storage) {Almacenamiento};

    \node [block, dashed, fill=gray, anchor=center, text width=7cm, align=center] at ($(system)!.5!(storage)$) {Conversión};

    % Connect nodes
    \draw [->] (input) -- node {$E_i$} (system);
    \draw [->] (system) -- node {$E_o$} (output);
    \draw [->] (storage) -| (outfork);
    \draw [->] (infork) |- (storage);
  \end{tikzpicture}
  \caption[Diagrama de bloques.]{Diagrama de bloques del sistema
    propuesto de almacenamiento energético, como ejemplo de código
    TikZ insertado directamente en el texto (ver archivo
    \code{intro.tex}, línea \number\inputlineno).}
  \label{fig:diagbloques}
\end{figure}

Nótese que un diagrama de bloques es distinto a un diagrama de etapas.
En general para este informe se prefiere el diagrama de bloques, pues
el diagrama de etapas tiene una connotación de documentación de
bitácora, que no es el objetivo de este informe.  Aquí se debe
explicar cómo reproducir los resultados a que finalmente se llegó, en
vez de explicar el proceso circunstancial y particular que usted
siguió para hacerlo; es decir, el proceso que usted siguió
posiblemente requirió pruebas fallidas y otras exploraciones que no
viene al caso explicar aquí (pero que usted sí documenta en su
bitácora, que es otro documento aparte), sino que aquí lleva al lector
por la ruta de exito directamente.

Usualmente este diagrama y su breve explicación dictan cuál será la
estructura del resto del documento, pues usted en el
\capref{ch:marco} deberá explicar los fundamentos teóricos que
cada bloque en esa solución requiere, y en el
\capref{ch:solucion} presentará una versión con mayor detalle de
esa solución, en donde ya considera lo expuesto en el marco teórico.



\section{Objetivos y estructura del documento}

\index{objetivos}
Esta plantilla LaTeX tiene como objetivo simplificar la construcción del
documento de tesis, presentando ejemplo de figuras y \tablas, así como otorgar
una plataforma de compilación en GNU/Linux que simplifique la administración de
todo el documento.

La última sección de la introducción usualmente sí tiene un título estandar que
es ``Objetivos y estructura del documento'', donde se presentan \emph{en prosa}
los objetivos general y específicos que ha tenido el proyecto de tesis,
así como la estructura de la tesis (por ejemplo, ``en el siguiente capítulo se
esbozan los fundamentos teóricos necesarios para explicar en el
\capref{ch:solucion} la propuesta realizada$\ldots$''

%%% Local Variables: 
%%% mode: latex
%%% TeX-master: "main"
%%% End: 

  \chapter{Marco teórico}
\label{ch:marco}

En este capítulo se presentan los conceptos teóricos que subyacen la propuesta de desarrollo de un conjunto de flujos de trabajo para la implementación de software 
a bordo de computadoras de guía, navegación y control espacial. La información expuesta se deriva tanto de conocimientos propios como información bibliográfica.

\section{Estimación}
La estimación implica el uso de modelos matemáticos y algoritmos para calcular las variables de estado del sistema. Estas variables son esenciales para comprender 
el comportamiento del sistema y para tomar decisiones informadas sobre su control. La estimación puede realizarse de dos maneras:

\begin{itemize}
    \item Lazo abierto: En este enfoque, se utilizan modelos de estimación predefinidos sin retroalimentación, lo que significa que las estimaciones no se ajustan en función
    de las mediciones reales.
    \item Lazo cerrado: Este método ajusta las estimaciones en función de las mediciones reales y las salidas del sistema, lo que permite una mayor precisión y adaptabilidad.
\end{itemize}

Esta es crucial en aplicaciones donde las mediciones directas son difíciles o costosas de obtener, por ejemplo en los sistemas hidráulicos, la estimación de variables de 
estado permite optimiza el rendimiento y la eficiencia del sistema, asegurando que se mantengan las condiciones deseadas a pesar de las perturbaciones externas o errores en las 
mediciones \cite{Merchn2019EvaluacinDM}. La estimación es un componente clave en los sistemas de control, ya que facilita la comprensión y el manejo de sistemas complejos. 
Su implementación permite una operación más eficiente y efectiva, mejorando su capacidad de respuesta ante diversa condiciones operativas \cite{Mesa2020EstimacinDV}.

\section{Control}
Como se mencionó anteriormente la estimación es un componente clave en los sistemas de control, ya que este se enfoca en el desarrollo y diseño de sistemas capaces de regular 
y controlar variables de un proceso de manera autónoma. Estos sistemas utilizan sensores, actuadores y algoritmos de control para mantener las variables de interés dentro de los 
rangos permitidos, mejorando de esta forma la eficiencia, precisión y confiabilidad de los procesos. Su aplicación abarca desde sistemas espaciales hasta biorreactores y sistemas
de iluminación. 


\section{Procesadores embebidos}

Los procesadores embebidos son microprocesadores especializados en tareas dentro de un sistema más complejo. A diferencia de los procesadores de propósito general, estos están 
optimizados para ofrecer eficiencia energética, un tamaño compacto y costo reducido. Algunas de las características de los procesadores embebidos se presentan a continuación:

\begin{itemize}
    \item Integración de periféricos: Incorporan periféricos específicos de la aplicación en un único chip, incluyendo temporizadores, puertos de entrada/salida y controladores 
    de memoria.
    \item Arquitecturas de bajo Consumo: Diseñados para maximizar la duración de la batería en dispositivos portátiles, lo que es esencial para la operatividad de dispositivos 
    móviles.
    \item Tamaño compacto: Su diseño permite reducir costos y facilitar la integración en espacios limitados, lo que los hace ideales para aplicaciones donde el espacio es 
    crítico.
    \item Capacidad de respuesta en tiempo real: Pueden responder a eventos externos de manera predecible y determinista, lo que es crucial en aplicaciones que requieren una respuesta rápida y precisa.
\end{itemize}

\subsection{Cortex-A9}

Los procesadores embebidos basados en la arquitectura ARM Cortex-A9 se utilizan en aplicaciones de alto rendimiento y capacidades avanzadas de procesamiento.
Aunque esta arquitectura no es un procesador embebido, sino más bien una familia de núcleos de procesador diseñado por ARM Holdings, los SoC que incorporan
estos núcleos han demostrado ser una solución popular para aplicaciones embebidas \cite{Schwiegelshohn2014DesignOA}. Algunas de sus características son : 

\begin{itemize}
    \item Arquitectura de 32 bits basada en ARMv7-A.
    \item Alto rendimiento adecuado para aplicaciones exigentes como sistemas operativos embebidos, procesamiento multimedia y gráficos.
    \item Características avanzadas como unidades de coma flotante, unidades de procesamiento NEON para procesamiento multimedia y soporte para virtualización.
\end{itemize}

Algunos SoC que incorporan núcleos Cortex-A9 son:

\begin{itemize}
    \item Nvidia Tegra 3: Combina cuatro núcleos Cortex-A9 y una GPU.
    \item Texas Instruments OMAP 4: Familia de SoC que combina núcleos Cortex-A9 y DSP.
    \item Xilinx Zynq-7000: Integra núcleos Cortex-A9 con lógica programable FPGA.
\end{itemize}

\subsection{Tarjeta de desarrollo ZedBoard}

La ZedBoard es una tarjeta de desarrollo basada en el Xilinx Zynq-7000 que como se mencionó anteriormente integra núcleos Cortex-A9 con la lógica programable 
para Field Programmable Gate Array,por sus siglas en ingles (FPGA). Esta plataforma es ideal para prototipar aplicaciones en el ámbito de sistemas embebidos. La tabla \ref{tab:zedboard} 
resume las especificaciones que posee la tarjeta de desarrollo ZedBoard.


\begin{figure}[h!]
    \centering
    \includegraphics[width=0.8\textwidth]{fig/teorico/zedboard_raw.png}
    \caption{Tarjeta de desarrollo Zedboard}
    \label{fig:zedboard_raw_info}
\end{figure}

Como se pudo observar en \ref{fig:zedboard_raw_info}, esta tarjeta de desarrollo cuenta con los puertos de conexión que se mencionan en la Tabla \ref{tab:puertos_yocto}: 

\begin{table}[htbp!]
    \caption{Puertos de entrada y salida de la plataforma de desarrollo Zedbord}
    \label{tab:puertos_yocto}
    \resizebox{\textwidth}{!}{%
    \begin{tabular}{|l|l|ll}
    \hline
    Identificador & Descripción                     & \multicolumn{1}{l|}{Identificador} & \multicolumn{1}{l|}{Descripción}      \\ \hline
    a             & Conector JTAG Xilinx            & \multicolumn{1}{l|}{l}             & \multicolumn{1}{l|}{Pulsadores}       \\ \hline
    b & Entrada de voltaje y   interruptor de encendido & \multicolumn{1}{l|}{m} & \multicolumn{1}{l|}{LEDs}                                 \\ \hline
    c             & USB-JTAG para   programación    & \multicolumn{1}{l|}{n}             & \multicolumn{1}{l|}{Interruptores}    \\ \hline
    d             & Puertos de audio                & \multicolumn{1}{l|}{o}             & \multicolumn{1}{l|}{Pantalla OLED}    \\ \hline
    e & Puerto de ethernet                              & \multicolumn{1}{l|}{p} & \multicolumn{1}{l|}{Botones de   programación y reinicio} \\ \hline
    f             & Puerto HDMI (Salida   de video) & \multicolumn{1}{l|}{q}             & \multicolumn{1}{l|}{Conectores Pmod}  \\ \hline
    g & Puerto VGA (Salida de   video)                  & \multicolumn{1}{l|}{r} & \multicolumn{1}{l|}{USB OTG para   perifericos}           \\ \hline
    h             & Puerto XADC                     & \multicolumn{1}{l|}{s}             & \multicolumn{1}{l|}{USB UART}         \\ \hline
    i             & Jumpers de   configuración      & \multicolumn{1}{l|}{t}             & \multicolumn{1}{l|}{Memoria DDR3}     \\ \hline
    j             & Conector FCM                    & \multicolumn{1}{l|}{u}             & \multicolumn{1}{l|}{Dispositivo Zynq} \\ \hline
    k             & Entrada para tarjeta   SD       &                                    &                                       \\ \cline{1-2}
    \end{tabular}%
    }
\end{table}

\begin{table}[h!]
    \caption{Especificaciones generales de la tarjeta de desarrollo ZeadBoard}
    \label{tab:zedboard}
    \resizebox{\textwidth}{!}{%
    \begin{tabular}{|l|l|}
    \hline
    \multicolumn{1}{|c|}{\textbf{Especificación}} & \multicolumn{1}{c|}{\textbf{Detalles}} \\ \hline
    \textbf{Procesador}             & Xilinx Zynq-7000 (XC7Z020)                         \\ \hline
    \textbf{Núcleos de Procesador}  & ARM Cortex-A9 de doble núcleo                      \\ \hline
    \textbf{Memoria DDR3}           & 512 MB                                             \\ \hline
    \textbf{Memoria Flash}          & 256 MB QSPI                                        \\ \hline
    \textbf{Almacenamiento}         & Tarjeta SD de 4 GB                                 \\ \hline
    \textbf{Conectividad}           & Ethernet (10/100/1000 Mbps), USB OTG 2.0, USB-UART \\ \hline
    \textbf{Salidas de Video}       & HDMI (1080p), VGA de 8 bits, OLED 128x32           \\ \hline
    \textbf{Audio}                  & Códec de audio I2S                                 \\ \hline
    \textbf{Puertos GPIO}           & 54 pines GPIO                                      \\ \hline
    \textbf{Interfaz de JTAG}       & Soporte para programación y depuración             \\ \hline
    \textbf{Dimensiones}            & 10.2 cm x 6.4 cm                                   \\ \hline
    \textbf{Fuente de Alimentación} & 5V a través de conector de alimentación            \\ \hline
    \textbf{Sistema Operativo}      & Soporte para Linux y otros sistemas embebidos      \\ \hline
    \textbf{Expansión}              & Conectores Pmod y FMC para módulos adicionales     \\ \hline
    \end{tabular}%
    }
\end{table}

Además de esto algunas especificaciones de la plataforma de desarrollo son mencionadas en \ref{tab:zedboard}. Por otro lado la ZedBoard es una plataforma de desarrollo altamente versátil que se destaca por su capacidad para ejecutar sistemas operativos como Linux, lo que la convierte en una opción ideal para proyectos de diseño y desarrollo de sistemas embebidos. Además de su compatibilidad con Linux, como se mencionó anteriormente la ZedBoard cuenta con especificaciones técnicas que la hacen destacar en el ámbito del desarrollo. Entre ellas se incluyen un procesador ARM Cortex-A9, una FPGA. 

\section{Marcos de trabajo}

Los marcos de trabajo en sistemas embebidos son conjuntos de herramientas y bibliotecas que facilitan el desarrollo de aplicaciones en estos 
sistemas. Estos proporcionan una estructura que permite abordar los desafíos específicos que presentan los sistemas embebidos.

Los sistemas embebidos interactúan con su entorno físico, lo que requiere un diseño que no solo considere los resultados de las operaciones, 
sino también el cumplimiento de plazos y restricciones específicas. En este contexto, las propiedades no funcionales, como el consumo energético, 
la latencia, la fiabilidad y el manejo de recursos, son críticas para el diseño y optimización del rendimiento general del sistema \cite{Marugn2017SimulacinYV}. Los frameworks 
juegan un papel fundamental al proporcionar herramientas y bibliotecas predefinidas, permitiendo a los desarrolladores centrarse en la lógica de la 
aplicación en lugar de lidiar con los detalles de bajo nivel del hardware, lo que acelera el proceso de desarrollo y reduce la posibilidad de errores. 
Ejemplos de frameworks populares en sistemas embebidos incluyen Robot Operating System (ROS), utilizado en aplicaciones de robótica, y FreeRTOS, 
un sistema operativo de tiempo real diseñado para microcontroladores y sistemas embebidos \cite{HerreraLpez2023EntornoDT}.

\subsection{YOCTO}\label{subsec:yocto}

Yocto es un marco de trabajo o bien del inglés (framework) popular utilizado en el desarrollo de sistemas embebidos, especialmente en la creación de distribuciones de Linux 
personalizadas para hardware específico. Yocto utiliza un proceso de construcción cruzada, lo que significa que el código se compila en una plataforma diferente 
a la que se ejecutará, permitiendo que el código se optimice para el hardware específico del sistema embebido \cite{Leppakoski2013FrameworkFI}.

\begin{figure}[h!]
    \centering
    \includegraphics[width=0.5\textwidth]{fig/teorico/Flujo de trabajo de yocto.pdf}
    \caption{Flujo de trabajo Yocto Project}
    \label{fig:yocto_project_workflow}
\end{figure}

Una de las principales ventajas de Yocto es su flexibilidad en la configuración del sistema, permitiendo a los desarrolladores seleccionar paquetes específicos, 
configurar opciones de compilación y personalizar el sistema operativo según sus necesidades. Además, Yocto fomenta la reutilización de código a través de capas, 
que son colecciones de recetas, configuraciones y parches que se pueden agregar o eliminar fácilmente del flujo de trabajo de construcción \cite{Leppakoski2013FrameworkFI}.

\section{MATLAB}
MATLAB es un software de cálculo técnico desarrollado por MathWorks, ampliamente utilizado en diversas áreas de la ciencia y la ingeniería. Proporciona un entorno interactivo para el desarrollo de algoritmos, análisis de datos, visualización y cálculo numérico. Su facilidad para trabajar con vectores y matrices lo distingue de otros sistemas de cálculo \cite{PealozaLuna2022SimulacinDU}. Es comúnmente utilizado para simular sistemas eléctricos, como convertidores DC/DC, y realizar análisis numéricos en problemas complejos \cite{OrdezGarca2022MatlabCU}. 

MATLAB juega un papel crucial en el desarrollo de sistemas de control aeroespaciales, facilitando la simulación, modelado y control de vehículos aéreos no tripulados (UAV) y otros sistemas relacionados.MATLAB, junto con Simulink, permite el modelado cinemático y dinámico de UAVs. Esto incluye la programación y control de sus movimientos, como cabeceo y guiñada, utilizando motores de corriente directa y encoders ópticos para medir su posición \cite{Senz2020LaboratorioPE} \cite{ChvezGudio2023DesarrolloYC}.

\subsection{Simulink}

Como se mencionó anteriormente Simulink, es un entorno de simulación y diseño gráfico que forma parte del software MATLAB. Se utiliza principalmente para modelar, simular y analizar sistemas dinámicos, especialmente aquellos que involucran componentes eléctricos, mecánicos y de control, algunas  de las características principales de Simulink son:

\begin{itemize}
    \item Modelado Gráfico
    \item Simulación en Tiempo real
    \item Integración con MATLAB
    \item Diseño de Controladores
    \item Análisis de Sistemas Dinámicos
\end{itemize}

De esta forma podemos ver que simulink es una herramienta que permite a los usuarios crear modelos visuales de sistemas complejos utilizando bloques representativos, facilitando así su diseño y comprensión. Ofrece simulaciones en tiempo real, esenciales para evaluar el comportamiento de sistemas en ingeniería y control bajo diversas condiciones \cite{PealozaLuna2022SimulacinDU}. Su integración con MATLAB potencia las capacidades de análisis y programación, permitiendo un análisis más profundo y la personalización de simulaciones \cite{Daza2021PlataformaDP}. Simulink se aplica en diversas áreas como la ingeniería eléctrica, mecánica, robótica y diseño de sistemas de control, siendo especialmente útil para diseñar y probar controladores (como PID y Fuzzy), analizar sistemas dinámicos y desarrollar prototipos rápidos para sistemas embebidos \cite*{CardozoSarmiento2019SimulationOI}.

\section{Transformación de modelo a modelo}

La transformación de modelo a modelo se refiere a un proceso en el que un modelo se convierte en otro, manteniendo la esencia de su estructura y funcionalidad, 
pero adaptándose a nuevas necesidades o contextos. Este concepto es fundamental en la Ingeniería de Software, especialmente dentro de la Arquitectura Dirigida 
por Modelos (MDA), donde se busca facilitar la interoperabilidad y la portabilidad de sistemas a través de la transformación de modelos independientes de la computación 
(CIM) a modelos independientes de la plataforma (PIM) y viceversa.

\begin{itemize}
    \item Modelos de Datos a Modelos de Aplicación:
    \item Modelos de Negocio a Modelos de Implementación
    \item Modelos UML a Código Fuente
\end{itemize}

Para efectos de este trabajo el área de interés serán la transformación de UML a Código Fuente.

\subsection{MATLAB Embedded Coder}

El MATLAB Embedded Coder se adapta a esta definición de transformación de modelo a modelo, ya que permite a los usuarios generar código C y C++ a partir de modelos 
Simulink. Esto es especialmente útil en el desarrollo de sistemas embebidos, donde se requiere que los modelos de alto nivel se transformen en código 
que pueda ser ejecutado en hardware específico. Esta herramienta facilita la implementación de algoritmos y sistemas de control, asegurando que el modelo original 
se traduzca eficazmente en un formato que pueda ser utilizado en entornos de producción.

\subsection{MATLAB Simulink Coder}

Simulink Coder es una herramienta del entorno MATLAB/Simulink que permite generar automáticamente código C y C++ a partir de modelos gráficos, facilitando la implementación de algoritmos en hardware o software. Sus características incluyen la generación de código automática, integración con diversas plataformas de hardware, optimización del rendimiento y soporte para modelos complejos, lo que la hace ideal para aplicaciones en control de sistemas, simulación y pruebas, y desarrollo ágil. En resumen, Simulink Coder es esencial para ingenieros que desean transformar modelos teóricos en aplicaciones prácticas, mejorando tanto el proceso de desarrollo como el rendimiento del producto final.
\newpage

\begin{figure}[h!]
    \centering
    \includegraphics[width=0.5\textwidth]{fig/teorico/Flujo de trabajo simulink coder.pdf}
    \caption{Flujo de trabajo Simulink Coder}
    \label{fig:Simulink_coder_workflow}
\end{figure}

Como se pudo observar en la Figura \ref{fig:Simulink_coder_workflow}, primeramente se debe de diseñar el modelo utilizando bloques de Simulink para representar el sistema o algoritmo a implementar, una vez generado el diagrama se deben de ajustar las configuraciones del modelo, incluyendo parámetros como el tipo de solución, la frecuencia de muestreo y las opciones de simulación, además de esto se deben de realizar simulaciones para verificar que el modelo funcione correctamente y cumpla con los requisitos especificados. 

Finalmente se debe de hacer uso de la herramienta Simulink Coder para generar automáticamente código C o C++ a partir del modelo validado. Adicionalmente a este código generado se pueden realizar configuraciones según se desea su ejecución, estas configuraciones son opciones adicionales para la generación del código, como optimización y estilo de codificación. Una vez aplicados todos los cambios necesarios se debe integrar el código generado en un entorno de desarrollo adecuado y realizar pruebas para asegurar que el código se comporta como se espera en el hardware objetivo.

\section{Código embebido}

El código embebido se refiere a un tipo de software diseñado para operar en dispositivos con recursos limitados, como microcontroladores y sistemas embebidos. 
Este código es fundamental en la programación de dispositivos electrónicos, permitiendo que estos realicen tareas específicas, como gestionar un sistema de 
automatización industrial o incluso operar en dispositivos móviles. Se caracteriza por su ejecución en dispositivos con recursos limitados, su capacidad 
para controlar dispositivos electrónicos, el uso de lenguajes de bajo nivel, la optimización de recursos y la necesidad de garantizar tiempos de respuesta 
determinísticos.

\section{Compilación Cruzada}

La compilación cruzada es un proceso clave en el desarrollo de software que permite compilar código fuente en un sistema operativo o arquitectura de hardware diferente al utilizado para el desarrollo. Este método es especialmente valioso en entornos como sistemas embebidos, donde la plataforma de destino no es adecuada para la compilación directa. Utiliza compiladores específicos, conocidos como compiladores cruzados, que generan código ejecutable para la plataforma de destino desde una plataforma de origen. Esto permite a los desarrolladores trabajar en sus máquinas locales, como Windows o Linux, mientras crean aplicaciones para dispositivos como microcontroladores o sistemas operativos variados.


Además de su uso en sistemas embebidos, la compilación cruzada es fundamental para el desarrollo multiplataforma, ya que facilita la creación de aplicaciones que funcionan en diferentes sistemas operativos sin necesidad de modificar el código base. Este enfoque no solo mejora la eficiencia del proceso de desarrollo y pruebas, sino que también evita la constante transferencia de código a la plataforma de destino. En resumen, la compilación cruzada es una técnica esencial en el desarrollo moderno de software, permitiendo a los desarrolladores abordar múltiples plataformas y arquitecturas con mayor facilidad.

\begin{figure}[h!]
    \centering
    \includegraphics[width=0.5\textwidth]{fig/teorico/Flujo de trabajo xcompiler.pdf}
    \caption{Diagrama de compilación cruzada}
    \label{fig:xcompile_workflow}
\end{figure}

Como se pudo observar en la Figura \ref{fig:xcompile_workflow}, el proceso de compilación de un programa en lenguaje C comienza con el código fuente, que es el conjunto de instrucciones escritas por el programador. A continuación, se utiliza el preprocesador, que lleva a cabo tareas esenciales como la inclusión de archivos y la expansión de macros, generando un archivo intermedio que está listo para ser compilado. Posteriormente, el compilador analiza este archivo intermedio y produce código objeto, una representación en lenguaje máquina que aún no está completamente vinculada. Este código objeto es luego procesado por el ensamblador, que lo convierte en código máquina, específico para la arquitectura del sistema objetivo.

\section{Contenedores}\label{sec:containers}

Docker ha revolucionado la forma en que se desarrollan y despliegan aplicaciones al ofrecer un entorno portátil y consistente. Gracias a su capacidad de empaquetar aplicaciones junto con todas sus dependencias, los desarrolladores pueden estar seguros de que su software funcionará de manera idéntica en cualquier entorno, ya sea local, en la nube o en producción. Esta portabilidad no solo ahorra tiempo en la configuración del entorno, sino que también reduce significativamente los problemas relacionados con "funciona en mi máquina". Además, el aislamiento que proporcionan los contenedores asegura que las aplicaciones operen sin interferencias, lo que es crucial para mantener la estabilidad y el rendimiento.

Por otro lado, la eficiencia de Docker es notable. A diferencia de las máquinas virtuales, los contenedores comparten el núcleo del sistema operativo, lo que permite un uso más optimizado de los recursos y un inicio casi instantáneo. Esto se traduce en una mayor agilidad y rapidez al escalar aplicaciones, ya que se pueden crear y gestionar múltiples instancias de contenedores con facilidad. La capacidad de versionar imágenes también es un gran beneficio, ya que permite a los equipos mantener un historial claro de cambios y revertir a versiones anteriores cuando sea necesario. En conjunto, estas características hacen de Docker una herramienta indispensable para la integración y entrega continua (CI/CD), mejorando significativamente los flujos de trabajo de desarrollo y despliegue.

\section{Protocolos de Comunicación}\label{sec:protocolos_de_comunicacion}

Los protocolos de comunicación son un conjunto de reglas y convenciones que permiten la transmisión de datos entre dispositivos en una red. Estos protocolos son esenciales para garantizar que los dispositivos puedan intercambiar información de manera efectiva y segura \cite{Eterovic2018AnlisisDP}.

\subsection{UART}

El Universal Asynchronous Receiver-Transmitter, por sus siglas en ingles (UART) es un protocolo de comunicación serial ampliamente utilizado para la transmisión de datos entre dispositivos. Su característica principal es que es asíncrono, lo que significa que no requiere una señal de reloj compartida entre el transmisor y el receptor.

\textbf{Características:}

\begin{itemize}
    \item Transmisión Asíncrona: No necesita sincronización de reloj, lo que simplifica su implementación y reduce la complejidad del sistema.
    \item Configuración Simple: Opera comúnmente con configuraciones de 8 bits de datos, 1 bit de parada y 1 bit de paridad opcional, facilitando su uso en diversas aplicaciones.
    \item Distancia de Comunicación: Es efectivo para distancias cortas, generalmente menos de 15 metros, debido a la posible degradación de la señal a medida que aumenta la distancia.
    \item Velocidad de Transmisión: Las tasas de baudios (baud rate) pueden variar desde 300 hasta 115200 bps o más, dependiendo del hardware y las condiciones del entorno.
\end{itemize}

\textbf{Aplicaciones:}

\begin{itemize}
    \item Comunicación entre microcontroladores.
    \item Interfaces para sensores y dispositivos periféricos.
    \item Envío de datos a través de puertos serie en computadoras y dispositivos embebidos.
\end{itemize}

\subsection{SSH}

El Secure Shell (SSH) es un protocolo de red que permite la administración segura de dispositivos y la transferencia de datos a través de redes inseguras. SSH proporciona autenticación y cifrado, garantizando que los datos transmitidos estén protegidos contra ataques maliciosos.

\textbf{Características:}

\begin{itemize}
    \item Cifrado: Utiliza algoritmos de cifrado robustos, como AES, para proteger la información durante su transmisión.
    \item Autenticación: Permite autenticación mediante contraseña o claves públicas, aumentando significativamente la seguridad en el acceso a los sistemas.
    \item Túneles Seguros: Facilita la creación de túneles seguros para otros protocolos, lo que permite la transferencia protegida de datos sensibles.
    \item Interfaz de Línea de Comando: Proporciona acceso remoto a la línea de comandos, permitiendo a los administradores gestionar sistemas sin necesidad de estar físicamente presentes.
\end{itemize}

\section{Computadoras de guía, navegación y control}

Las computadoras de guía, navegación y control son esenciales en diversas aplicaciones, particularmente en aviación, navegación marítima y vehículos autónomos. Su función principal radica en procesar datos de sensores para ofrecer información precisa que facilite la toma de decisiones en tiempo real. Estas tecnologías son fundamentales para garantizar la seguridad y eficiencia en el transporte moderno.

Existen tres tipos principales de computadoras en este ámbito: las computadoras de navegación, que determinan la posición y rumbo de embarcaciones o aeronaves mediante sistemas GPS e inerciales; las computadoras de control, que regulan el movimiento y estabilidad de los vehículos, como los controladores de vuelo; y las computadoras de guía, que ofrecen rutas óptimas a través de sistemas de navegación por satélite. Los componentes clave incluyen sensores que recogen datos del entorno, procesadores que realizan cálculos complejos y interfaces de usuario que permiten la interacción con el sistema.

Las aplicaciones de estas computadoras son variadas y críticas. En la aviación, se utilizan para el control de vuelo en aeronaves comerciales y militares. En la navegación marítima, aseguran rutas seguras para barcos. Además, son integradas en vehículos autónomos como coches y drones, permitiendo una navegación eficiente sin intervención humana. En conjunto, estas tecnologías no solo mejoran la seguridad, sino que también optimizan la eficacia del transporte actual.

\subsection{EXA ICEPS}\label{sec:exaiceps}
La computadora de vuelo EXA ICEPS (Integrated Control and Engine Performance System) es una tecnología avanzada diseñada para gestionar y optimizar el rendimiento del motor y otros sistemas críticos en aeronaves. Su integración de sistemas permite combinar el control del motor con diferentes funciones de la aeronave, lo que resulta en una gestión más eficiente y segura durante el vuelo.

Entre sus principales características se destacan el monitoreo en tiempo real, que proporciona datos sobre el rendimiento del motor, permitiendo a los pilotos tomar decisiones informadas. Además, la EXA ICEPS optimiza el rendimiento al maximizar la eficiencia del combustible y reducir las emisiones, contribuyendo así a operaciones más sostenibles. Sus funciones incluyen el control automático de los parámetros del motor y la facilitación del diagnóstico y mantenimiento predictivo, lo que ayuda a reducir costos operativos.

Este sistema ejemplifica cómo la tecnología moderna está revolucionando la aviación, mejorando tanto la seguridad como la eficiencia operativa. Al integrar múltiples funciones y proporcionar información crítica en tiempo real, la EXA ICEPS no solo optimiza el rendimiento de las aeronaves, sino que también promueve prácticas más sostenibles en la industria.
\section{Revisión literaria}
En los últimos años, las computadoras de guía, navegación y control han mostrado grandes avances en el desarrollo de sistemas autónomos.

\subsection{Desarrollo de sistemas de navegación}

En 2022, se presentó un sistema de planificación y control de navegación para vehículos autónomos en entornos urbanos. Este sistema permite la planificación 
de rutas basadas en la posición actual del vehículo y su destino, utilizando un controlador clásico que asegura el seguimiento de la trayectoria mediante 
odometría y correcciones visuales. Los resultados se simularon utilizando herramientas como ROS y Gazebo, lo que demuestra la viabilidad de estos sistemas en 
entornos complejos \cite{BarreraRamrez2022SistemaDP}. 

\subsection{Transformación de Lenguaje de Bloques a Código C}

La traducción de código de control de lenguaje de bloques a C implica un proceso de conversión donde cada bloque visual se asocia con una estructura de código 
en C. Esto se puede hacer utilizando herramientas de software que generan automáticamente el código C a partir de la lógica definida en el entorno de bloques. 
Este proceso no solo facilita la programación, sino que también permite la optimización del código generado para mejorar el rendimiento en sistemas de navegación 
autónoma.

\subsubsection{XOD}

XOD es un entorno de programación visual basado en bloques que permite a los usuarios crear programas para microcontroladores como Arduino. Este software 
genera automáticamente código en C++ a partir de la lógica definida en bloques. Los usuarios pueden conectar componentes gráficamente y, al finalizar, acceder 
al código generado, que es abierto y personalizable. XOD es gratuito y permite la creación de nuevos nodos para componentes específicos, lo que facilita la 
adaptación a diferentes proyectos \cite{Snchez2020ProgramacinDL}.

\subsubsection{Visual Microcontroller}
Este software proporciona un lenguaje de programación gráfico para microcontroladores, desarrollado en C\#. Utiliza una interfaz gráfica que permite a los 
usuarios diseñar diagramas que representan la lógica de control. El sistema compila el código a partir de diagramas gráficos, generando código intermedio 
en C antes de llegar al código hexadecimal necesario para la programación del microcontrolador \cite{Sacta2011DesarrolloDU}.


\subsubsection{LabVIEW}

LabVIEW es un entorno de desarrollo que utiliza un enfoque gráfico para la programación. Aunque es más conocido en el ámbito de la ingeniería, también 
permite la generación de código en C. LabVIEW facilita la creación de aplicaciones de control y adquisición de datos, y su capacidad para traducir 
diagramas de bloques a código C lo convierte en una opción útil para proyectos que requieren un control preciso de hardware.

\subsubsection{Simulink}
Como se mencionó anteriormente, Simulink, parte de MATLAB, proporciona un entorno gráfico para modelar, simular y analizar sistemas dinámicos. 
Permite a los usuarios crear modelos utilizando bloques y, posteriormente, generar código C automáticamente a partir de estos modelos. Esta 
herramienta es especialmente valiosa en aplicaciones de ingeniería donde se requiere un alto grado de precisión y control sobre el 
comportamiento del sistema.

\section{Avances recientes en GNCs }

En el marco del proyecto EROSS+ (European Robotic Orbital Support Services), se ha trabajado en el diseño de un sistema GNC altamente autónomo 
para misiones de servicio robótico en órbita. Este proyecto, que abarca desde 2021 hasta 2023, busca integrar técnicas avanzadas de navegación 
visual y control de cumplimiento para la captura y manipulación de satélites, mostrando un enfoque en la autonomía y la eficiencia operativa \cite{Casu2023EROSSPA}.

Otro desarrollo notable es el programa de NASA sobre GNC autónomo, que incluye sistemas para el transbordador espacial. Este programa se centra en la optimización 
de trayectorias de vuelo y la adaptación de sistemas GNC para diferentes condiciones de vuelo, lo que demuestra la importancia de la flexibilidad en el diseño de 
estos sistemas \cite{Bordano1991AutonomousGN}.

Además, la actividad VV4RTOS, apoyada por la Agencia Espacial Europea, se ha centrado en la verificación y validación de sistemas de control basados en optimización. 
Esto incluye el desarrollo de software GNC en tiempo real, lo que permite una validación más efectiva y segura de los sistemas diseñados \cite{Loureno2023VerificationV}.

\subsection{Programación de Sistemas GNC}
Los lenguajes de bloques, como Simulink, son comúnmente utilizados para diseñar y simular sistemas de control. Estos lenguajes permiten a los ingenieros visualizar 
el flujo de datos y las interacciones entre componentes de manera intuitiva. Sin embargo, la necesidad de traducir estos modelos a código C es crucial para su 
implementación en hardware real.

A pesar de los avances, existen desafíos significativos en la implementación de sistemas GNC. La variabilidad en los entornos operativos y la necesidad de adaptarse 
a condiciones cambiantes requieren algoritmos robustos y adaptativos. La optimización de estos sistemas es fundamental para asegurar su efectividad en misiones 
críticas.

Un estudio reciente sobre el sistema CubeNav destaca la importancia de desarrollar herramientas de análisis de navegación que faciliten las operaciones de GNC 
en misiones de CubeSats. Este enfoque busca reducir la curva de aprendizaje y minimizar errores humanos, lo que es esencial para misiones de bajo presupuesto 
y alta complejidad \cite{Loureno2023VerificationV}.
  \chapter{Plataforma de desarrollo}
\label{ch:especifico1}

En este capítulo se pretende identificar una plataforma de hardware para el desarrollo de un modelo de ingeniería de una computadora de guía, navegación y control espacial por sus siglas en ingles (GNC), para llevar a cabo este objetivo se plantean los requerimientos que se deben de tomar en cuenta para elegir una tarjeta de desarrollo que logre satisfacer las necesidades de este proyecto, seguido de esto se seleccionaran un grupo de tarjetas las cuales cumplan con los requerimientos previamente establecidos, estas serán comparadas para poder determinar cuál de las tarjetas de desarrollo seleccionadas puede cumplir de mejor forma la tarea seleccionada.

\section{Selección de la tarjeta de desarrollo}
    Para la selección de la tarjeta de desarrollo se partirá de la definición de los requerimientos de operación del sistema, esto tomando en cuenta las operaciones más comunes que realizan los sistemas GNC, una vez definidos los requerimientos, se seleccionaran al menos 3 tarjetas candidatas, esto con el fin de establecer los criterios de comparación para el desarrollo de una matriz de Pugh.

\subsection{Requerimientos de la aplicación}

Al elegir una tarjeta de desarrollo para un sistema de guía, navegación y control (GNC) en aplicaciones espaciales, se deben de tener en cuenta varios factores clave. Dentro de ellos se encuentran el procesamiento, más precisamente la capacidad de cálculo, ya que estos sistemas requieren de un procesamiento intensivo para los cálculos de trayectoria, estimación de estado y control. Seguido de esto se debe de considerar que sea un sistema de baja latencia, además que el mismo tenga soporte para sensores y actuadores para poder medir y controlar el sistema, además de esto los puertos de entrada y salida deben ofrecer la precisión necesaria ara leer los datos de los sensores que se conecten al mismo.

Por otro lado el sistema debe de contener capacidades de tiempo real estricto, ya que los sistemas GNC deben de tomar decisiones críticas en el momento requerido. Además de tener la capacidad de ejecutar un sistema operativo de tiempo real (RTOS) o bien Linux en tiempo real. También un aspecto importante a contener por la tarjeta de desarrollo es el consumo de energía, esto sin dejar de lado las capacidades de simulación y pruebas, ya que, en la interfaz de simulación la tarjeta se debe de poder conectar a un entorno de pruebas de hardware-in-the-loop por sus siglas en ingles (HIL), además de las capacidades de depuración y monitoreo.

Finalmente se deben de tomar en cuenta aspectos como lo son el Tamaño, peso y forma de la tarjeta buscando que las mismas contengan un tamaño compacto, ya que los sistemas espaciales siempre se deben de integrar en espacios reducidos y la resistencia del mismo.

\subsection{Tarjetas candidatas}
Bajo los requerimientos planteados anteriormente se eligieron las siguientes tarjetas de desarrollo, las mismas se presentarán con sus características.

\subsubsection{Xilinx ZCU102 Evaluation Kit}

La tarjeta de desarrollo ZCU102 contiene procesamiento basado en Zynq UltraScale+ MPSoC, el cual combina un procesador ARM Cortex-A53 de 64 bits con una FPGA de alto rendimiento, la cual es excelente para el procesamiento en tiempo real y algoritmos personalizados.

Por otro lado, esta ofrece una amplia gama de interfaces de comunicación como: PCIe, Ethernet, I2C, SPI, UART, GPIO. En cuanto a la eficiencia energética esta opción contiene mecanismos para el control de energía, además de ser compatible con entornos de simulación y tiene interfaces JTAG para una buena depuración. Finalmente es una tarjeta ampliamente utilizada en la industria en sistemas de prototipos avanzados y despliegue de HIL.

En síntesis esta opción ofrece un procesamiento potente y versátil además de ser excelente para desarrollar y escalar sistemas GNC complejos, por otro lado, es una tarjeta de desarrollo costosa.

\subsubsection{NVIDIA Jetson AGX Xavier}

Para el caso de la tarjeta AGX Xavier de NVIDIA incorpora una CPU ARM v8.2 de 64 bits y una GPU NVIDIA Volta, prestaciones las cuales se encargan de proporcionar un alto nivel de procesamiento de datos en paralelo especialmente utilizado para aplicaciones de visión por computador o inteligencia artificial para sistemas GNC. 

Las interfases de comunicación presentes en esta tarjeta son puertos I2C, SPIm UART y GPIO además de contener adicionalmente soporte nativo para cámaras y sensores de alta gama. Además presenta capacidades en tiempo real, ya que se puede implementar con el NVIDIA Jetpack SDK. 

En cuanto a la eficiencia energética, la misma posee un diseño optimizado para bajo consumo. Además de ser compatible con interfases de pruebas y simulación mediante el uso de entornos como Tensor RT y otras plataformas propietarias del desarrollador de la tarjeta de desarrollo.

Esta tarjeta es ideal para sistemas GNC con un procesamiento intensivo de datos ya sean de visión por computador o bien inteligencia artificial, por otro lado posee un desarrollo potente 
para el procesamiento de tares en paralelo o bien de aprendizaje reforzado, finalmente contiene un buen soporte para aplicaciones en tiempo real y simulaciones. Por otro lado contiene un 
bajo procesamiento lógico comparado con las FPGA para aplicaciones en tiempo real extremo, y esta tarjeta de desarrollo se encuentra más enfocada en la implementación de soluciones que
requieran inteligencia artificial.

\subsubsection{TMS320C6678 Development Kit}

La tarjeta TMS320C6678 es basada en un procesador para procesamiento digital de señales (DSP) de 8 núcleos, está enfocado a aplicaciones de procesamiento intensivo en tiempo real, soporta
interfases como lo son Ethernet, SPI, UARTm I2C y GPIO, además de tener opciones para expandir la conectividad de la misma, por otro lado como mencionamos anteriormente es uno de los sistemas
más optimizados para el procesamiento en tiempo real por medio de la plataforma propietaria TI RTOS.

Sobre la eficiencia energética, esta tarjeta de desarrollo ofrece herramientas específicas para la optimización del consumo de energía, haciéndola adecuada para entornos de consumo energético 
restringido, finalmente es compatible con Code Composer Studio, el cual es un entorno de desarrollo integrado que facilita las labores de integrar, simular y depurar el código implementado. Por tanto podemos decir que es una tarjeta de desarrollo muy adecuada para los sistemas de procesamiento de señales y control, tiene una gran capacidad para soportar aplicaciones industriarles y aeroespaciales. Por otro lado, podemos ver que es una plataforma menos flexible que una FPGA. 

\subsubsection{ZedBoard de Avnet}

En cuanto a procesamiento, para la tarjeta ZedBoard, tenemos que utiliza un procesador Xilinx Zynq-7000 APSoC el cual combina un procesador ARM Cortex-A9 dual core con una FPGA programable, de esta forma tomando el procesador ARM el cual es ideal para ejecutar algoritmos de control y  lógica de navegación en un entorno de RTOS o bien Linux, por otro lado la FPGA Zynq-7000 permite la ejecución de tareas de procesamiento paralelo en hardware como el filtrado de señales o algoritmos de estimación de estados, ofreciendo baja latencia y flexibilidad en tiempo real.

En cuanto a las interfases de entrada y salida, incluye varias opciones como lo son: GPIO, I2C, SPI, UART. Además de esto contiene puertos Ethernet, micro usb y HDMI los cuales resultan útiles para la comunicación externa y visualización de los sistemas de desarrollo.

La combinación de un procesador ARM con una FPGA permite un equilibrio en el consumo de energía, ya que la mayoría de tareas intensivas se pueden llevar a cabo en la FPGA y el SoC Zynq ofrece opciones de ahorro de energía lo cual siempre representa un beneficio para las aplicaciones embebidas. En cuanto a las pruebas y simulaciones, cuenta con entornos como Vivado y SDK de Xilinx esto con el fin de realizar simulaciones de HIL. Por otro lado ofrece aplicaciones para depuración como lo es JTAG para el monitoreo en tiempo real de las aplicaciones ejecutándose en la FPGA y en el procesador ARM.
\subsection{Criterios de comparación}

Una vez presentadas las tarjetas de desarrollo candidatas se procede con la definición de los criterios de comparación: para este caso los criterios a tomar en cuenta son los siguientes:

\begin{enumerate}
    \item Capacidad de procesamiento: La capacidad de procesamiento en dispositivos de desarrollo para sistemas GNC es crucial porque garantiza la ejecución en tiempo real de algoritmos complejos, como los de control y fusión de sensores, que son esenciales para la estabilidad y precisión del sistema. Permite procesar grandes volúmenes de datos de múltiples sensores de manera simultánea y rápida, ejecutar tareas en paralelo, y realizar cálculos intensivos como la planificación de trayectorias y control adaptativo. Además, un procesamiento robusto facilita la simulación HIL, asegurando pruebas y simulaciones realistas.

    \item Soporte para sensores y actuadores: El soporte para sensores y actuadores es esencial en dispositivos de desarrollo para sistemas GNC porque estos sistemas dependen de la entrada de múltiples sensores como acelerómetros, giroscopios, GPs, entre otros, para monitorear y estimar la posición, orientación y velocidad del vehículo en tiempo real. La capacidad de interactuar directamente con estos sensores, y con actuadores que ejecutan las acciones de control, es fundamental para garantizar la retroalimentación continua y precisa necesaria para el correcto funcionamiento del sistema GNC. Interfaces como I2C, SPI, UART, y GPIO permiten esta integración, asegurando un control eficiente y adaptable.

    \item Capacidad de trabajo en tiempo real: La capacidad de trabajo en tiempo real es vital en dispositivos de desarrollo para sistemas de guía, navegación y control (GNC) porque estos sistemas requieren respuestas inmediatas y precisas ante cambios en el entorno o en las condiciones del vehículo. Los algoritmos de control, como los de estabilidad y trayectoria, deben ejecutarse sin demoras para garantizar la seguridad y el rendimiento óptimo del sistema. Sin procesamiento en tiempo real, las decisiones de control podrían retrasarse, afectando la estabilidad y el control del vehículo, lo cual es crítico en aplicaciones como navegación autónoma o vuelo espacial.

    \item Consumo de energia: El consumo de energía es crucial en dispositivos de desarrollo para sistemas de guía, navegación y control (GNC), especialmente en aplicaciones espaciales o autónomas, donde los recursos energéticos son limitados. Un consumo eficiente permite que el sistema funcione de manera prolongada sin comprometer su rendimiento, maximizando la duración de la misión y asegurando que los componentes críticos, como sensores y actuadores, siempre reciban suficiente energía. Además, la gestión adecuada del consumo evita el sobrecalentamiento y prolonga la vida útil de los dispositivos, lo que es fundamental en entornos de operación prolongada o difíciles de acceder.

    \item Caracteristicas Fisicas: Las características físicas, como el tamaño, peso y forma, son importantes en dispositivos de desarrollo para sistemas de guía, navegación y control (GNC) porque estos sistemas suelen implementarse en entornos con restricciones de espacio y peso, como en vehículos aéreos, drones o satélites. Un dispositivo compacto y ligero facilita la integración en estos sistemas sin afectar su desempeño ni su capacidad de carga. Además, un diseño físico optimizado es clave para minimizar los efectos de vibraciones, choques o cambios de temperatura, asegurando un funcionamiento fiable en condiciones extremas.

    \item Costo: El costo es un factor importante en dispositivos de desarrollo para sistemas de guía, navegación y control (GNC) porque influye directamente en la viabilidad económica del proyecto, especialmente en fases de prototipado o prueba. Un dispositivo con un costo adecuado permite realizar iteraciones y pruebas sin superar el presupuesto, facilitando el acceso a tecnologías avanzadas sin comprometer la calidad. Además, un costo equilibrado permite escalar el proyecto o implementar múltiples sistemas de prueba, optimizando el desarrollo sin sacrificar funcionalidad o capacidad técnica.

    \item Escalabilidad del sistema: 
    La escalabilidad del sistema es crucial en dispositivos de desarrollo para sistemas de guía, navegación y control (GNC) porque permite adaptar el hardware y software a medida que el proyecto crece en complejidad o requisitos técnicos. Un dispositivo escalable facilita la integración de nuevos sensores, algoritmos más avanzados o mayores capacidades de procesamiento sin necesidad de cambiar completamente la plataforma. Esto ahorra tiempo y costos, además de asegurar que el sistema pueda evolucionar para cumplir con las demandas de futuras fases del desarrollo o nuevas aplicaciones, manteniendo la flexibilidad y la eficiencia.

\end{enumerate}
\newpage

\section{Matriz de Pugh}

% Please add the following required packages to your document preamble:
% \usepackage{graphicx}
\begin{table}[h!]
    \centering
    \caption{Matriz de Pugh para seleccionar la tarjeta de desarrollo que mejor de adapte a los requerimientos del proyecto}
    \label{tab:Pug_tarjetas_desarrollo}
    \resizebox{\columnwidth}{!}{%
    \begin{tabular}{lccccc}
    \hline
    \multicolumn{1}{|l|}{Criterios} & \multicolumn{1}{l|}{Peso} & \multicolumn{1}{l|}{ZCU102} & \multicolumn{1}{l|}{AGX Xavier} & \multicolumn{1}{l|}{TMS320C6678} & \multicolumn{1}{l|}{Zedboard} \\ \hline
    \multicolumn{1}{|l|}{Capacidad de   procesamiento} & \multicolumn{1}{c|}{15} & \multicolumn{1}{c|}{15} & \multicolumn{1}{c|}{15} & \multicolumn{1}{c|}{8} & \multicolumn{1}{c|}{10} \\ \hline
    \multicolumn{1}{|l|}{Soporte para   sensores} & \multicolumn{1}{c|}{15} & \multicolumn{1}{c|}{15} & \multicolumn{1}{c|}{15} & \multicolumn{1}{c|}{15} & \multicolumn{1}{c|}{15} \\ \hline
    \multicolumn{1}{|l|}{Soporte para   actuadores} & \multicolumn{1}{c|}{15} & \multicolumn{1}{c|}{15} & \multicolumn{1}{c|}{15} & \multicolumn{1}{c|}{15} & \multicolumn{1}{c|}{15} \\ \hline
    \multicolumn{1}{|l|}{Soporte de   sistemas de tiempo real} & \multicolumn{1}{c|}{20} & \multicolumn{1}{c|}{20} & \multicolumn{1}{c|}{20} & \multicolumn{1}{c|}{15} & \multicolumn{1}{c|}{20} \\ \hline
    \multicolumn{1}{|l|}{Caracteristicas   Fisicas} & \multicolumn{1}{c|}{10} & \multicolumn{1}{c|}{4} & \multicolumn{1}{c|}{7} & \multicolumn{1}{c|}{10} & \multicolumn{1}{c|}{10} \\ \hline
    \multicolumn{1}{|l|}{Costo de la   tarjeta} & \multicolumn{1}{c|}{15} & \multicolumn{1}{c|}{7} & \multicolumn{1}{c|}{10} & \multicolumn{1}{c|}{10} & \multicolumn{1}{c|}{15} \\ \hline
    \multicolumn{1}{|l|}{Escalabilidad   del sistema} & \multicolumn{1}{c|}{10} & \multicolumn{1}{c|}{10} & \multicolumn{1}{c|}{6} & \multicolumn{1}{c|}{6} & \multicolumn{1}{c|}{10} \\ \hline
     & \multicolumn{1}{l}{} & \multicolumn{1}{l}{} & \multicolumn{1}{l}{} & \multicolumn{1}{l}{} & \multicolumn{1}{l}{} \\ \cline{1-1} \cline{3-6} 
    \multicolumn{1}{|l|}{Suma general} & \multicolumn{1}{l|}{} & \multicolumn{1}{c|}{86} & \multicolumn{1}{c|}{88} & \multicolumn{1}{c|}{79} & \multicolumn{1}{c|}{95} \\ \cline{1-1} \cline{3-6} 
    \multicolumn{1}{|l|}{Posicion} & \multicolumn{1}{l|}{} & \multicolumn{1}{c|}{3} & \multicolumn{1}{c|}{2} & \multicolumn{1}{c|}{4} & \multicolumn{1}{c|}{1} \\ \cline{1-1} \cline{3-6} 
    \end{tabular}%
    }
    \end{table}

    Como se pudo observar en la Tabla \ref{tab:Pug_tarjetas_desarrollo}, un claro ganador segun los requerimientos establecidos para este proyecto ha sido la tarjeta de desarrollo zedbaord ya que es la mejor opcion en cuanto a caracteristicas como lo son la capacidad de procesamiento y \dots

\section{Plataforma seleccionada}

Como se pudo observar en la Tabla \ref{tab:Pug_tarjetas_desarrollo}, la tarjeta de desarrollo seleccionada fue la Trajeta Zedboard de Avnet. La ZedBoard es una tarjeta de desarrollo basada en el SoC (System-on-Chip) Xilinx Zynq-7000. Diseñada para aplicaciones de desarrollo en sistemas embebidos y de procesamiento de señales, la ZedBoard combina la potencia de un procesador ARM con la flexibilidad de una FPGA (Field Programmable Gate Array), proporcionando una plataforma versátil para la investigación, el desarrollo y la prueba de diversas aplicaciones, incluidas las de guía, navegación y control (GNC).

\subsection{Especificaciones principales}

Como se mencionó en el capítulo \ref{ch:marco} en la Tabla \ref{tab:zedboard} y anteriormente en este capítulo la tarjeta en cuestión presenta las siguientes especificaciones principales.

\subsubsection{Procesador y FPGA}
SoC Xilinx Zynq-7000: La ZedBoard integra un procesador ARM Cortex-A9 dual-core junto con una FPGA programable de la serie 7-Series.
ARM Cortex-A9: Ofrece un rendimiento de procesamiento general que puede ejecutar sistemas operativos como Linux o FreeRTOS, lo que es útil para tareas de control y procesamiento de datos.
FPGA: La FPGA proporciona capacidad para implementar lógica personalizada, lo que permite el desarrollo de algoritmos específicos en hardware para procesamiento en tiempo real y alta velocidad.
\subsubsection{Interfaces de E/S}
GPIO: General Purpose Input/Output, permite la interacción con una amplia gama de periféricos y sensores.
I2C, SPI, UART: Protocolos de comunicación estándar que facilitan la integración con diversos dispositivos de sensor y actuadores.
Ethernet: Conectividad de red para comunicación y transmisión de datos.
USB: Puertos USB para conexión de dispositivos externos y almacenamiento.
HDMI: Salida de video para visualización de datos y control gráfico.
JTAG: Para depuración y programación de la FPGA y el procesador ARM.
\subsubsection{Memoria}
RAM: Incluye memoria DDR3 SDRAM para el procesador y la FPGA, proporcionando espacio suficiente para la ejecución de sistemas operativos y algoritmos complejos.
Flash: Memoria flash para almacenamiento de configuraciones y datos persistentes.
\subsubsection{Alimentación y Consumo de Energía}
La ZedBoard se alimenta típicamente a través de una entrada de 5V, con un diseño que optimiza el consumo energético para aplicaciones de desarrollo. Sin embargo, el consumo real depende del uso de la FPGA y el procesador.
\subsubsection{Tamaño y Factor de Forma}
Dimensiones: Aproximadamente 15.24 cm x 22.86 cm (6 x 9 pulgadas), lo que la hace adecuada para prototipos sin ser excesivamente grande.
Diseño: Compacta pero con suficiente espacio para interfaces y módulos adicionales.
\subsubsection{Capacidades de Desarrollo}
Entornos de Desarrollo: Compatible con Xilinx Vivado Design Suite y SDK, proporcionando herramientas avanzadas para diseño, simulación, y depuración.
Ejemplos de Aplicaciones: Adecuada para aplicaciones que requieren procesamiento en paralelo, desarrollo de sistemas embebidos, y prueba de algoritmos de control.

\section{Reflexión final}

Como se pudo observar a lo largo de este capítulo se analizaron los requerimientos de hardware que se deben de tomar en cuenta para el desarrollo de este proyecto, seguido de esto se tomaron cuatros tarhjetas de desarrolo con el fin de elegir entre las que presentaran las prestaciones adecuadas para la tarea a realizar. Por un lado se tenian tarjetas muy potentes como xxxx y xxxx y por otro lado se tenian tarjetas de grandes dimenciones como la xxx y xxxx. Segun los parametros definidos se elige la tarjeta xxxx con numero de parte xxx para continuar con el desarrollo de este proyecto.
  \chapter{Solución propuesta}
\label{ch:especifico2}

Como se pudo obervar en el capitulo \ref{ch:especifico1}, se realizo la seleccion de la tarjeta de desarrollo Zedboard para el desarrollo del proyecto, en este capitulo se pretende establecer flujos de trabajo para el prototipado de algoritmos de control de orientacion y navegacion para aplicaciones espaciales. Esto mediante el uso de Matlab Simulink para tomar un caso de uso como ejemplo, seguido de esto convertir el codigo por medio de la trasnformacion de modelo de simulink a modelo de codigo C, con el objetivo de poder embedder el codigo en una imagen minima por medio del flujo de trabajo de Yocto Project y finalmente probar el mismo en la tarjeta de desarrollo zedboard y asi comparar los resultados obtenidos y el tiempo de ejecucion que llevo la tarea en el computador y en el sistema embebido.

\section{GNC embebido software workflow}

Anteriormente definimos un flujo de trabajo para poder establecer el prototipado de algoritmos de control, orientacion y navegacion para aplicaciones espaciales, el mismo prentendemos logaralo mediante la seleccion de un caso de estudio en matlab simulink el cual se mostrara en el desarrollo de este capitulo

\section{Flujo de trabajo de la aplicación Model 2 Model Transformation}
\subsection{Caso de estudio}
\section{Flijo de Trabajo Herramienta desarrollada por mi persona}
\subsection{Caso de estudio}
\section{Reflexion final}
  \chapter{Caso de estudio IMU y PID}
\label{ch:especifico3}

\section{Caso de estudio 2 - IMU}

Como caso de estudio de implementación de una Unidad de medición inercial, IMU por sus siglas en Ingles, se siguió el uso del modelo que se presenta en \cite{mathworks2024imu}. Este ejemplo muestra cómo generar y fusionar datos de sensores IMU usando MATLAB Simulink. Permitiendo modelar con precisión el comportamiento de un acelerómetro, un giroscopio y un magnetómetro, además de poder fusionar sus salidas para calcular la orientación.

Una IMU es un grupo de sensores que incluye un acelerómetro para medir aceleración y un giroscopio para medir velocidad angular. Frecuentemente, también se incluye un magnetómetro para medir el campo magnético de la Tierra. Cada uno de estos tres sensores produce una medición de tres ejes, constituyendo una medición de 9 ejes en total. Ademas de esto un Sistema de Referencia de Actitud y Rumbo (AHRS, por sus siglas en inglés) toma las lecturas de sensores de 9 ejes y calcula la orientación del dispositivo. Esta orientación se da en relación con el marco NED, donde N es la dirección del Norte Magnético. El bloque AHRS en Simulink logra esto usando una estructura de filtro de Kalman indirecto \cite{mathworks2024imu}.

\subsection{Implementación en MATLAB Simulink}

\begin{figure}[h!]
    \centering
    \includegraphics[width=0.8\textwidth]{fig/especifico_2/154140ZedBoard.png}
    \caption{Diagrama completo del caso de estudio 2 - IMU \cite{mathworks2024imu}}
    \label{fig:caso_de_estudio_2_IMU}
\end{figure}


Como se puede observar en la Figura \ref{fig:caso_de_estudio_2_IMU}, este es el caso de estudio que se propone en \cite{mathworks2024imu}, a este caso de estudio se le deben de realizar unas modificaciones de acuerdo al funcionamiento deseado que se tiene para este caso de estudio, siempre generando datos en el ámbito de simulación en MATLAB para luego contrastar los mismos con los datos obtenidos en la ejecución del modelo en la tarjeta de desarrollo seleccionada.

\subsection{Bloques utilizados para la implementación}

Los bloques utilizados se obtienen en la librería de bloques de MATLAB Simulink. A continuación se muestran los bloques requeridos, asi como la configuración de los mismos para la correcta operación del modelo. La implementación del sistema se divide en dos partes, el primer parte se encarga de generar los archivos necesarios para la operación del sistema mientras que la segunda parte del sistema se encarga de leer los archivos con los datos y generar los dos archivos de salida del programa.

\subsubsection{Sistema para la generación de archivos}

Este sistema es el encargado de generar los archivos de entrada, estos mismos contienen los datos de tiempo y valores para la correcta implementación del sistema

\begin{figure}[htbp]
    \centering
    \begin{subfigure}[b]{0.45\textwidth}
        \centering
        \includegraphics[width=\textwidth]{fig/Capitulo5/Caso_de_estudio_IMU/Generador_de_archivos/libreria_de_bloques_aceleracion_lineal.png}
        \caption{Librería de bloques - Aceleración Lineal}
        \label{fig:lib_bloques_linear_acceleration}
    \end{subfigure}
    \hfill
    \begin{subfigure}[b]{0.45\textwidth}
        \centering
        \includegraphics[width=\textwidth]{fig/Capitulo5/Caso_de_estudio_IMU/Generador_de_archivos/configuracion_bloque_aceleracion_lineal.png}
        \caption{Configuración del bloque aceleración lineal}
        \label{fig:lib_bloques_config_linear_acceleration}
    \end{subfigure}
    \caption{Bloque para la aceleración lineal}
    \label{fig:linear_accel_block_simulink}
\end{figure}


\begin{figure}[htbp]
    \centering
    \begin{subfigure}[b]{0.45\textwidth}
        \centering
        \includegraphics[width=\textwidth]{fig/Capitulo5/Caso_de_estudio_IMU/Generador_de_archivos/libreria_de_bloques_constante_velocidad_angular.png}
        \caption{Librería de bloques - Velocidad Angular}
        \label{fig:lib_bloques_angular_velocity}
    \end{subfigure}
    \hfill
    \begin{subfigure}[b]{0.45\textwidth}
        \centering
        \includegraphics[width=\textwidth]{fig/Capitulo5/Caso_de_estudio_IMU/Generador_de_archivos/configuracion_bloque_velocidad_angular.png}
        \caption{Configuración del bloque velocidad angular}
        \label{fig:lib_bloques_config_angular_velocity}
    \end{subfigure}
    \caption{Bloque para la velocidad angular}
    \label{fig:angular_velocity_block_simulink}
\end{figure}


\begin{figure}[htbp]
    \centering
    % Primera imagen
    \begin{subfigure}[b]{0.35\textwidth}
        \centering
        \includegraphics[width=\textwidth]{fig/Capitulo5/Caso_de_estudio_IMU/Generador_de_archivos/libreria_de_bloques_IMU.png}
        \caption{Librería de bloques - IMU}
        \label{fig:lib_bloques_IMU}
    \end{subfigure}
    \hfill
    % Segunda imagen
    \begin{subfigure}[b]{0.45\textwidth}
        \centering
        \includegraphics[width=\textwidth]{fig/Capitulo5/Caso_de_estudio_IMU/Generador_de_archivos/configuracion_parametros_IMU_01.png}
        \caption{Configuración de parámetros 1}
        \label{fig:parametros_IMU_01}
    \end{subfigure}
    \hfill
    % Tercera imagen
    \begin{subfigure}[b]{0.45\textwidth}
        \centering
        \includegraphics[width=\textwidth]{fig/Capitulo5/Caso_de_estudio_IMU/Generador_de_archivos/configuracion_parametros_IMU_02.png}
        \caption{Configuración de parámetros 2}
        \label{fig:parametros_IMU_02}
    \end{subfigure}
    \hfill
    % Cuarta imagen
    \begin{subfigure}[b]{0.45\textwidth}
        \centering
        \includegraphics[width=\textwidth]{fig/Capitulo5/Caso_de_estudio_IMU/Generador_de_archivos/configuracion_parametros_IMU_03.png}
        \caption{Configuración de parámetros 3}
        \label{fig:parametros_IMU_03}
    \end{subfigure}
    \hfill
    % Quinta imagen
    \begin{subfigure}[b]{0.45\textwidth}
        \centering
        \includegraphics[width=\textwidth]{fig/Capitulo5/Caso_de_estudio_IMU/Generador_de_archivos/configuracion_parametros_IMU_04.png}
        \caption{Configuración de parámetros 4}
        \label{fig:parametros_IMU_04}
    \end{subfigure}

    \caption{Bloque para la simulación del comportamiento de la IMU}
    \label{fig:arreglo_imu}
\end{figure}


\begin{figure}[htbp]
    \centering
    \begin{subfigure}[b]{0.35\textwidth}
        \centering
        \includegraphics[width=\textwidth]{fig/Capitulo5/Caso_de_estudio_IMU/Generador_de_archivos/libreria_de_bloques_subsistema_integracion_velocidad_angular.png}
        \caption{Librería de bloques - Integrador}
        \label{fig:lib_bloques_integrador}
    \end{subfigure}
    \hfill
    \begin{subfigure}[b]{0.45\textwidth}
        \centering
        \includegraphics[width=\textwidth]{fig/Capitulo5/Caso_de_estudio_IMU/Generador_de_archivos/configuracion_integrador_velocidad_angular.png}
        \caption{Configuración del bloque velocidad angular}
        \label{fig:config_bloques_integrador}
    \end{subfigure}
    \caption{Bloque para la integración de la velocidad angular}
    \label{fig:integration_for_angular_velocity}
\end{figure}


%Missing image for required block information
%\begin{figure}[htbp]
%    \centering
%    \begin{subfigure}[b]{0.35\textwidth}
%        \centering
%        \includegraphics[width=\textwidth]{fig/Capitulo5/Caso_de_estudio_IMU/Generador_de_archivos/libreria_de_bloques_to_file.png}
%        \caption{Librería de bloques - Guardar en archivo}
%        \label{fig:lib_bloques_to_file_IMU}
%    \end{subfigure}
%    \hfill
%    \begin{subfigure}[b]{0.45\textwidth}
%        \centering
%        \includegraphics[width=\textwidth]{fig/Capitulo5/Caso_de_estudio_IMU/Generador_de_archivos/}
%        \caption{}
%        \label{fig:}
%    \end{subfigure}
%    \caption{}
%    \label{fig:}
%\end{figure}


\subsubsection{Sistema para la lectura e interpretación de los archivos generados previamente}

\begin{figure}[htbp]
    \centering
    \begin{subfigure}[b]{0.35\textwidth}
        \centering
        \includegraphics[width=\textwidth]{fig/Capitulo5/Caso_de_estudio_IMU/Generador_de_salidas/libreia_de_bloques_from_file.png}
        \caption{Librería de bloques - Leer de archivo}
        \label{fig:lib_bloques_from_file_IMU}
    \end{subfigure}
    \hfill
    \begin{subfigure}[b]{0.45\textwidth}
        \centering
        \includegraphics[width=\textwidth]{fig/Capitulo5/Caso_de_estudio_IMU/Generador_de_salidas/libreia_de_bloques_from_file.png}
        \caption{Configuración del bloque encargado de la lectura de archivos}
        \label{fig:config_from_file_IMU}
    \end{subfigure}
    \caption{Bloque para la lectura de archivos}
    \label{fig:read_from_file}
\end{figure}


\begin{figure}[htbp]
    \centering
    \begin{subfigure}[b]{0.35\textwidth}
        \centering
        \includegraphics[width=\textwidth]{fig/Capitulo5/Caso_de_estudio_IMU/Generador_de_salidas/libreia_de_bloques_suma.png}
        \caption{Librería de bloques - Suma}
        \label{fig:lib_bloques_add_IMU}
    \end{subfigure}
    \hfill
    \begin{subfigure}[b]{0.45\textwidth}
        \centering
        \includegraphics[width=\textwidth]{fig/Capitulo5/Caso_de_estudio_IMU/Generador_de_salidas/configuracion_bloque_suma.png}
        \caption{Configuración del bloque encargado de la suma de señales}
        \label{fig:config_add_IMU}
    \end{subfigure}
    \caption{Bloque para la suma de señales}
    \label{fig:add_of_some_signals}
\end{figure}


\begin{figure}[htbp]
    \centering
    % Primera imagen
    \begin{subfigure}[b]{0.35\textwidth}
        \centering
        \includegraphics[width=\textwidth]{fig/Capitulo5/Caso_de_estudio_IMU/Generador_de_salidas/libreira_de_bloques_sensor_AHRS.png}
        \caption{Librería de bloques - AHRS}
        \label{fig:lib_bloques_AHRS}
    \end{subfigure}
    \hfill
    % Segunda imagen
    \begin{subfigure}[b]{0.45\textwidth}
        \centering
        \includegraphics[width=\textwidth]{fig/Capitulo5/Caso_de_estudio_IMU/Generador_de_salidas/configuracion_AHRS_01.png}
        \caption{Configuración de parámetros 1}
        \label{fig:parametros_AHRS_01}
    \end{subfigure}
    \hfill
    % Tercera imagen
    \begin{subfigure}[b]{0.45\textwidth}
        \centering
        \includegraphics[width=\textwidth]{fig/Capitulo5/Caso_de_estudio_IMU/Generador_de_salidas/configuracion_AHRS_01.png}
        \caption{Configuración de parámetros 2}
        \label{fig:parametros_AHRS_02}
    \end{subfigure}
    \hfill
    % Cuarta imagen
    \begin{subfigure}[b]{0.45\textwidth}
        \centering
        \includegraphics[width=\textwidth]{fig/Capitulo5/Caso_de_estudio_IMU/Generador_de_salidas/configuracion_AHRS_01.png}
        \caption{Configuración de parámetros 3}
        \label{fig:parametros_AHRS_03}
    \end{subfigure}

    \caption{Bloque para la simulación del comportamiento de la IMU}
    \label{fig:arreglo_AHRS}
\end{figure}

\begin{figure}[htbp]
    \centering
    \begin{subfigure}[b]{0.35\textwidth}
        \centering
        \includegraphics[width=\textwidth]{fig/Capitulo5/Caso_de_estudio_IMU/Generador_de_salidas/libreria_bloque__rad_2_deg.png}
        \caption{Librería de bloques - Conversor de radianes a grados}
        \label{fig:lib_bloques_R2D}
    \end{subfigure}
    \hfill
    \begin{subfigure}[b]{0.45\textwidth}
        \centering
        \includegraphics[width=\textwidth]{fig/Capitulo5/Caso_de_estudio_IMU//Generador_de_salidas/libreria_bloque__rad_2_deg.png}
        \caption{Configuración del bloque conversor de radianes a grados}
        \label{fig:conf_bloques_R2D}
    \end{subfigure}
    \caption{Bloque para convertir de Radianes a grados}
    \label{fig:bloques_R2D}
\end{figure}


\begin{figure}[htbp]
    \centering
    \begin{subfigure}[b]{0.35\textwidth}
        \centering
        \includegraphics[width=\textwidth]{fig/Capitulo5/Caso_de_estudio_IMU/Generador_de_salidas/libreria_bloque_de_funcion.png}
        \caption{Librería de bloques - Función}
        \label{fig:lib_bloques_func}
    \end{subfigure}
    \hfill
    \begin{subfigure}[b]{0.45\textwidth}
        \centering
        \includegraphics[width=\textwidth]{fig/Capitulo5/Caso_de_estudio_IMU/Generador_de_salidas/libreria_bloque_de_funcion.png}
        \caption{}
        \label{fig:config_bloques_func}
    \end{subfigure}
    \caption{Bloque para aplicar una función implementada mediante código}
    \label{fig:bloques_func}
\end{figure}

\subsection{Resultados de la simulación}

\subsection{Implementación en la Tarjeta de desarrollo mediante EmbedSynthGNC}

Para la implementación en la tarjeta de desarrollo ZedBoard se ejecuta el flujo de trabajo que se muestra en la sección \ref{}. El mismo es representado mediante el diagrama que se muestra en la Figura \ref{}.



\subsection{Resultados de la implementación}

\section{Caso de estudio 3 - PID}

\subsection{Implementación en MATLAB Simulink}

\subsection{Implementación en la Tarjeta de desarrollo mediante EmbedSynthGNC}

  \chapter{Conclusiones}

El proyecto de desarrollar flujos de trabajo para la implementación de software a bordo de computadoras de guía, navegación y control espacial ha logrado avances significativos en varios aspectos clave. Primero, la identificación de una plataforma de hardware adecuada para el desarrollo de un modelo de ingeniería de una computadora de navegación espacial ha sido un paso crucial. Esta plataforma no solo ha proporcionado un marco sólido para el diseño y la prueba, sino que también ha permitido optimizar los recursos y reducir los tiempos de desarrollo, lo que a su vez ha aumentado la eficiencia general del proyecto.

El establecimiento de flujos de trabajo para el prototipado de algoritmos de control de orientación y navegación con hardware en el loop ha sido otro logro notable. Estos flujos de trabajo han permitido una integración más fluida entre el software y el hardware, lo que ha mejorado significativamente la precisión y la estabilidad de los sistemas de navegación espacial. Además, la capacidad de probar y ajustar estos algoritmos en tiempo real ha acelerado el proceso de desarrollo y ha reducido el riesgo de errores críticos.

La evaluación de los casos de uso de una computadora de navegación y control espacial a través de la implementación de una aplicación de referencia demostrativa, específicamente el caso de la Unidad de Medida Inercial (IMU), ha proporcionado valiosos insights prácticos. Esta aplicación ha demostrado la viabilidad y el rendimiento de los flujos de trabajo desarrollados, validando la efectividad de los algoritmos y la plataforma de hardware seleccionada. Este enfoque práctico ha permitido identificar y abordar desafíos reales, asegurando que el sistema final sea robusto y confiable.

En resumen, el proyecto ha alcanzado sus objetivos específicos de manera satisfactoria, lo que ha sentado las bases para futuras innovaciones en el campo de la navegación y control espacial. La combinación de una plataforma de hardware adecuada, flujos de trabajo eficientes para el prototipado y la evaluación práctica de los casos de uso ha asegurado que el proyecto esté bien posicionado para enfrentar los desafíos complejos de la exploración espacial. Estos logros no solo mejoran las capacidades actuales sino que también abren caminos para investigaciones y aplicaciones futuras en este campo.


  %----------------------------------------------------------------------------
  % literature in bibtex way:
  % \bibliographystyle{sty/plainurl} % for english documents
  % \bibliography{literatura}
  % literature in biblatex/biber way
  \printbibliography[title={Bibliografía},heading=bibintoc]
  %----------------------------------------------------------------------------

  %----------------------------------------------------------------------------
  \appendix
  %----------------------------------------------------------------------------

  \chapter{Métricas de comparación de señales}
\label{apx:comparacion_de_sennales_programa}

\begin{lstlisting}[language=python, caption={Métricas de comparación de señales}, label=lst:comparacion_de_senales_sim_vs_exp]
    import h5py
    import scipy.io
    import matplotlib.pyplot as plt

    #Cargar los archivos .mat
    #En la linea 7 se debe de colocar la direccion del archivo de simulacion generado por MATLAB
    with h5py.File(r"archivo_simulado.mat", 'r') as archivo:
        print("Claves del archivo:", list(archivo.keys()))
        senal_simulada = archivo[list(archivo.keys())[0]][:]


    #En la linea 12 se debe de colocar la direccion del archivo generado al ejecutar el programa en el ZedBoard
    senal_experimental = scipy.io.loadmat('archivo_experimental.mat')

    vector1 = senal_simulada  
    vector2 = senal_experimental  

    #Calculo del Error promedio absoluto
    mae = np.mean(np.abs(vector1 - vector2))
    print(f"Error Promedio Absoluto (MAE): {mae}")
    
    #Calculo de la raiz del error cuadratico medio
    rmse = np.sqrt(np.mean((vector1 - vector2) ** 2))
    print(f"Raíz del Error Cuadrático Medio (RMSE): {rmse}")
    
    #Calculo del error cuadratico medio
    mse = np.mean((vector1 - vector2) ** 2)
    print("Error Cuadrático Medio (MSE):", mse)
\end{lstlisting}

  %----------------------------------------------------------------------------
  \backmatter
  %----------------------------------------------------------------------------

  \printindex                % insert index into document. Don't forget to call
                             % "makeindex filename" first.
\end{document}
