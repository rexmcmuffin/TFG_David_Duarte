\chapter{Marco teórico}
\label{ch:marco}

En este capítulo se presentan los conceptos teóricos que subyacen la propuesta de desarrollo de un conjunto de flujos de trabajo para la implementación de software 
a bordo de computadoras de guía, navegación y control espacial. La información expuesta se deriva tanto de conocimientos propios como información bibliográfica.

\section{Estimación}
La estimación implica el uso de modelos matemáticos y algoritmos para calcular las variables de estado del sistema. Estas variables son esenciales para comprender 
el comportamiento del sistema y para tomar decisiones informadas sobre su control. La estimación puede realizarse de dos maneras:

\begin{itemize}
    \item Lazo abierto: En este enfoque, se utilizan modelos de estimación predefinidos sin retroalimentación, lo que significa que las estimaciones no se ajustan en función
    de las mediciones reales.
    \item Lazo cerrado: Este método ajusta las estimaciones en función de las mediciones reales y las salidas del sistema, lo que permite una mayor precisión y adaptabilidad.
\end{itemize}

Esta es crucial en aplicaciones donde las mediciones directas son difíciles o costosas de obtener, por ejemplo en los sistemas hidráulicos, la estimación de variables de 
estado permite optimiza el rendimiento y la eficiencia del sistema, asegurando que se mantengan las condiciones deseadas a pesar de las perturbaciones externas o errores en las 
mediciones \cite{Merchn2019EvaluacinDM}. La estimación es un componente clave en los sistemas de control, ya que facilita la comprensión y el manejo de sistemas complejos. 
Su implementación permite una operación más eficiente y efectiva, mejorando su capacidad de respuesta ante diversa condiciones operativas \cite{Mesa2020EstimacinDV}.

\section{Control}
Como se mencionó anteriormente la estimación es un componente clave en los sistemas de control, ya que este se enfoca en el desarrollo y diseño de sistemas capaces de regular 
y controlar variables de un proceso de manera autónoma. Estos sistemas utilizan sensores, actuadores y algoritmos de control para mantener las variables de interés dentro de los 
rangos permitidos, mejorando de esta forma la eficiencia, precisión y confiabilidad de los procesos. Su aplicación abarca desde sistemas espaciales hasta biorreactores y sistemas
de iluminación. 


\section{Procesadores embebidos}

Los procesadores embebidos son microprocesadores especializados en tareas dentro de un sistema más complejo. A diferencia de los procesadores de propósito general, estos están 
optimizados para ofrecer eficiencia energética, un tamaño compacto y costo reducido. Algunas de las características de los procesadores embebidos se presentan a continuación:

\begin{itemize}
    \item Integración de periféricos: Incorporan periféricos específicos de la aplicación en un único chip, incluyendo temporizadores, puertos de entrada/salida y controladores 
    de memoria.
    \item Arquitecturas de bajo Consumo: Diseñados para maximizar la duración de la batería en dispositivos portátiles, lo que es esencial para la operatividad de dispositivos 
    móviles.
    \item Tamaño compacto: Su diseño permite reducir costos y facilitar la integración en espacios limitados, lo que los hace ideales para aplicaciones donde el espacio es 
    crítico.
    \item Capacidad de respuesta en tiempo real: Pueden responder a eventos externos de manera predecible y determinista, lo que es crucial en aplicaciones que requieren una respuesta rápida y precisa.
\end{itemize}

\subsection{Cortex-A9}

Los procesadores embebidos basados en la arquitectura ARM Cortex-A9 se utilizan en aplicaciones de alto rendimiento y capacidades avanzadas de procesamiento.
Aunque esta arquitectura no es un procesador embebido, sino más bien una familia de núcleos de procesador diseñado por ARM Holdings, los SoC que incorporan
estos núcleos han demostrado ser una solución popular para aplicaciones embebidas \cite{Schwiegelshohn2014DesignOA}. Algunas de sus características son : 

\begin{itemize}
    \item Arquitectura de 32 bits basada en ARMv7-A.
    \item Alto rendimiento adecuado para aplicaciones exigentes como sistemas operativos embebidos, procesamiento multimedia y gráficos.
    \item Características avanzadas como unidades de coma flotante, unidades de procesamiento NEON para procesamiento multimedia y soporte para virtualización.
\end{itemize}

Algunos SoC que incorporan núcleos Cortex-A9 son:

\begin{itemize}
    \item Nvidia Tegra 3: Combina cuatro núcleos Cortex-A9 y una GPU.
    \item Texas Instruments OMAP 4: Familia de SoC que combina núcleos Cortex-A9 y DSP.
    \item Xilinx Zynq-7000: Integra núcleos Cortex-A9 con lógica programable FPGA.
\end{itemize}

\subsection{Tarjeta de desarrollo ZedBoard}

La ZedBoard es una tarjeta de desarrollo basada en el Xilinx Zynq-7000 que como se menciono anteriormente integra núcleos Cortex-A9 con la lógica programable 
para Field Programmable Gate Array (FPGA). Esta plataforma es ideal para prototipar aplicaciones en el ámbito de sistemas embebidos. La tabla \ref{tab:zedboard} 
resume las especificaciones que posee la tarjeta de desarrollo ZedBoard.


\begin{table}[h!]
    \caption{Especificaciones generales de la tarjeta de desarrollo ZeadBoard }
    \label{tab:zedboard}
    \resizebox{\textwidth}{!}{%
    \begin{tabular}{|l|l|}
    \hline
    \multicolumn{1}{|c|}{\textbf{Especificación}} & \multicolumn{1}{c|}{\textbf{Detalles}} \\ \hline
    \textbf{Procesador}             & Xilinx Zynq-7000 (XC7Z020)                         \\ \hline
    \textbf{Núcleos de Procesador}  & ARM Cortex-A9 de doble núcleo                      \\ \hline
    \textbf{Memoria DDR3}           & 512 MB                                             \\ \hline
    \textbf{Memoria Flash}          & 256 MB QSPI                                        \\ \hline
    \textbf{Almacenamiento}         & Tarjeta SD de 4 GB                                 \\ \hline
    \textbf{Conectividad}           & Ethernet (10/100/1000 Mbps), USB OTG 2.0, USB-UART \\ \hline
    \textbf{Salidas de Video}       & HDMI (1080p), VGA de 8 bits, OLED 128x32           \\ \hline
    \textbf{Audio}                  & Códec de audio I2S                                 \\ \hline
    \textbf{Puertos GPIO}           & 54 pines GPIO                                      \\ \hline
    \textbf{Interfaz de JTAG}       & Soporte para programación y depuración             \\ \hline
    \textbf{Dimensiones}            & 10.2 cm x 6.4 cm                                   \\ \hline
    \textbf{Fuente de Alimentación} & 5V a través de conector de alimentación            \\ \hline
    \textbf{Sistema Operativo}      & Soporte para Linux y otros sistemas embebidos      \\ \hline
    \textbf{Expansión}              & Conectores Pmod y FMC para módulos adicionales     \\ \hline
    \end{tabular}%
    }
    \end{table}



\section{Marcos de trabajo}

Los marcos de trabajo en sistemas embebidos son conjuntos de herramientas y bibliotecas que facilitan el desarrollo de aplicaciones en estos 
sistemas. Estos proporcionan una estructura que permite abordar los desafíos específicos que presentan los sistemas embebidos.

Los sistemas embebidos interactúan con su entorno físico, lo que requiere un diseño que no solo considere los resultados de las operaciones, 
sino también el cumplimiento de plazos y restricciones específicas. En este contexto, las propiedades no funcionales, como el consumo energético, 
la latencia, la fiabilidad y el manejo de recursos, son críticas para el diseño y optimización del rendimiento general del sistema \cite{Marugn2017SimulacinYV}. Los frameworks 
juegan un papel fundamental al proporcionar herramientas y bibliotecas predefinidas, permitiendo a los desarrolladores centrarse en la lógica de la 
aplicación en lugar de lidiar con los detalles de bajo nivel del hardware, lo que acelera el proceso de desarrollo y reduce la posibilidad de errores. 
Ejemplos de frameworks populares en sistemas embebidos incluyen Robot Operating System (ROS), utilizado en aplicaciones de robótica, y FreeRTOS, 
un sistema operativo de tiempo real diseñado para microcontroladores y sistemas embebidos \cite{HerreraLpez2023EntornoDT}.

\subsection{YOCTO}

Yocto es un marco de trabajo (framework) popular utilizado en el desarrollo de sistemas embebidos, especialmente en la creación de distribuciones de Linux 
personalizadas para hardware específico. Yocto utiliza un proceso de construcción cruzada, lo que significa que el código se compila en una plataforma diferente 
a la que se ejecutará, permitiendo que el código se optimice para el hardware específico del sistema embebido \cite{Leppakoski2013FrameworkFI}.

Una de las principales ventajas de Yocto es su flexibilidad en la configuración del sistema, permitiendo a los desarrolladores seleccionar paquetes específicos, 
configurar opciones de compilación y personalizar el sistema operativo según sus necesidades. Además, Yocto fomenta la reutilización de código a través de capas, 
que son colecciones de recetas, configuraciones y parches que se pueden agregar o eliminar fácilmente del flujo de trabajo de construcción \cite{Leppakoski2013FrameworkFI}.

\section{Transformación de modelo a modelo}

La transformación de modelo a modelo se refiere a un proceso en el que un modelo se convierte en otro, manteniendo la esencia de su estructura y funcionalidad, 
pero adaptándose a nuevas necesidades o contextos. Este concepto es fundamental en la Ingeniería de Software, especialmente dentro de la Arquitectura Dirigida 
por Modelos (MDA), donde se busca facilitar la interoperabilidad y la portabilidad de sistemas a través de la transformación de modelos independientes de la computación 
(CIM) a modelos independientes de la plataforma (PIM) y viceversa.

\begin{itemize}
    \item Modelos de Datos a Modelos de Aplicación:
    \item Modelos de Negocio a Modelos de Implementación
    \item Modelos UML a Código Fuente
\end{itemize}

Para efectos de este trabajo el area de interes seran la transformación de UML a Código Fuente.

\subsection{MATLAB Embedded Coder}

El MATLAB Embedded Coder se adapta a esta definición de transformación de modelo a modelo, ya que permite a los usuarios generar código C y C++ a partir de modelos 
Simulink y Stateflow. Esto es especialmente útil en el desarrollo de sistemas embebidos, donde se requiere que los modelos de alto nivel se transformen en código 
que pueda ser ejecutado en hardware específico. Esta herramienta facilita la implementación de algoritmos y sistemas de control, asegurando que el modelo original 
se traduzca eficazmente en un formato que pueda ser utilizado en entornos de producción

\section{Código embebido}

El código embebido se refiere a un tipo de software diseñado para operar en dispositivos con recursos limitados, como microcontroladores y sistemas embebidos. 
Este código es fundamental en la programación de dispositivos electrónicos, permitiendo que estos realicen tareas específicas, como gestionar un sistema de 
automatización industrial o incluso operar en dispositivos móviles. Se caracteriza por su ejecución en dispositivos con recursos limitados, su capacidad 
para controlar dispositivos electrónicos, el uso de lenguajes de bajo nivel, la optimización de recursos y la necesidad de garantizar tiempos de respuesta 
determinísticos.

\section{Revisión literaria}
\section{Estado del arte}