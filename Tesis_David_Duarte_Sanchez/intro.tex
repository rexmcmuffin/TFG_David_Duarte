%% ---------------------------------------------------------------------------
%% intro.tex
%%
%% Introduction
%%
%% $Id: intro.tex 1477 2010-07-28 21:34:43Z palvarado $
%% ---------------------------------------------------------------------------

\chapter{Introducción}
\label{chp:intro}

En la \nt{introducción} deben quedar completamente claros los siguientes
aspectos, cuyo significado depende del tipo concreto de tesis:

\begin{compactitem}
\item Contexto
\item Antecedentes
\item Problema concreto
\item Esbozo de solución
\item Objetivos y estructura
\end{compactitem}

Una buena introducción debe lograr que el lector tenga interés de leer el resto
del tesis.

Es recomendable dividir la tesis en secciones, nombradas cada una de acuerdo a
su contenido. \textbf{Jamás} utilice los nombres de la guía como
``\emph{Problema existente e importancia de su solución}'', sino algo como ``La
deforestación en Costa Rica'' o lo que se adecúe a su problema en particular.

Recuerde que en español solo la primera letra del título va en mayúscula
(exceptuando nombres propios, por supuesto).
%
Algunos recursos adicionales a esta guía los encuentra en \cite{AlvaradoWeb}.


\section{El cambio climático y la electrónica}
\label{sec:contexto}

El contexto corresponde al entorno donde se desarrolla el proyecto de
tesis, que puede ser el área general de aplicación, un dominio de
problemas, etc.

De nuevo, no use un título genérico como ``Contexto'', sino algo
asociado directamente a su trabajo.


\section{Antecedentes}

Si su proyecto se circunscribe en otro proyecto mayor, en el que han
participado otros estudiantes de grado y postgrados, y ya existen
tesis o artículos publicados, en esta sección se hace una breve reseña
de esos trabajos previos, con el objetivo de contextualizar en dónde
calza concretamente el trabajo actual dentro de ese otro proyecto
mayor.  Por ejemplo, Fulano en~\cite{Fulano21} exploró si un diodo
puede funcionar como fuente de energía infinita, hipótesis que no
logró comprobar.

En proyectos relativamente aislados, no es necesaria esta sección.

Dependiendo de cada trabajo concreto, esta sección puede desplazarse a
otro lugar dentro de la introducción donde tenga más sentido, pero
usualmente se encuentra aquí justo antes de presentar el problema
técnico concreto tratado en su proyecto.

\section{La disipación de energía en el reactor 42}

En esta sección usted debe exponer su problema concreto.  Debe enlazar
el contexto general, expuesto en las secciones anteriores, con el
problema concreto que este trabajo resuelve.

Al final de esta sección, el problema concreto se sintetiza usualmente
en una frase de planteamiento del problema de ingeniería o pregunta
generadora de la investigación de ingeniería. Esta frase o pregunta
debería ser una consecuencia a la que se llega después de realizar el
desarrollo del contexto.  Si el problema es de caracter científico,
aquí puede plantearse la hipótesis de la investigación científica.

Del planteamiento del problema se deriva cuál es el objetivo del
trabajo en particular, que a su vez debe conducir al lector de forma
natural al esbozo de la solución del problema a tratar en este
informe.

\section{Sistema de almacenamiento energético}

Después de las secciones anteriores ya ha guiado al lector hasta este
punto en donde solo resta presentar una propuesta general de solución
del problema técnico concreto.

Para aclarar la solución se hace uso de un diagrama de bloques (ver
\figref{fig:diagbloques}) o diagrama de flujo general, es decir,
desde un nivel de abstracción muy alto, donde no sea necesario entrar
en detalles técnicos, porque aun no han sido expuestos.

\begin{figure}[htb]

  %% Este es un ejemplo de figura TIKZ incrustada directamente
  %% en el la figura, pero es muchísimo más recomendable poner este
  %% código, tal y como se explica en el siguiente capítulo, en un archivo
  %% aparte.  Como archivo aparte se puede compilar la figura una sola vez
  %% para que quede disponible en el directorio fig/.  Eso es más rápido de
  %% compilar posteriormente.
  %%
  %%
  %% Se coloca este ejemplo aquí porque muchas personas están usando
  %% Overleaf, y allí puede tener sentido tener las figuras
  %% directamente en el código del informe, aunque esto tardará más en
  %% compilar e incluso puede llevar a Overleaf a pasarse del tiempo
  %% disponible.
  
  \centering
  \tikzstyle{block} = [draw, rectangle, inner sep=6pt]
  \begin{tikzpicture}[>=latex,auto,node distance=2cm]
    \node [block](system) {Sistema};
    \node [coordinate, left=of system] (infork) {};
    \node [coordinate, left=of infork] (input) {};
    \node [coordinate, right=of system] (outfork) {};
    \node [coordinate, right=of outfork] (output) {};
    \node [block, below=of system] (storage) {Almacenamiento};

    \node [block, dashed, fill=gray, anchor=center, text width=7cm, align=center] at ($(system)!.5!(storage)$) {Conversión};

    % Connect nodes
    \draw [->] (input) -- node {$E_i$} (system);
    \draw [->] (system) -- node {$E_o$} (output);
    \draw [->] (storage) -| (outfork);
    \draw [->] (infork) |- (storage);
  \end{tikzpicture}
  \caption[Diagrama de bloques.]{Diagrama de bloques del sistema
    propuesto de almacenamiento energético, como ejemplo de código
    TikZ insertado directamente en el texto (ver archivo
    \code{intro.tex}, línea \number\inputlineno).}
  \label{fig:diagbloques}
\end{figure}

Nótese que un diagrama de bloques es distinto a un diagrama de etapas.
En general para este informe se prefiere el diagrama de bloques, pues
el diagrama de etapas tiene una connotación de documentación de
bitácora, que no es el objetivo de este informe.  Aquí se debe
explicar cómo reproducir los resultados a que finalmente se llegó, en
vez de explicar el proceso circunstancial y particular que usted
siguió para hacerlo; es decir, el proceso que usted siguió
posiblemente requirió pruebas fallidas y otras exploraciones que no
viene al caso explicar aquí (pero que usted sí documenta en su
bitácora, que es otro documento aparte), sino que aquí lleva al lector
por la ruta de exito directamente.

Usualmente este diagrama y su breve explicación dictan cuál será la
estructura del resto del documento, pues usted en el
\capref{ch:marco} deberá explicar los fundamentos teóricos que
cada bloque en esa solución requiere, y en el
\capref{ch:solucion} presentará una versión con mayor detalle de
esa solución, en donde ya considera lo expuesto en el marco teórico.



\section{Objetivos y estructura del documento}

\index{objetivos}
Esta plantilla LaTeX tiene como objetivo simplificar la construcción del
documento de tesis, presentando ejemplo de figuras y \tablas, así como otorgar
una plataforma de compilación en GNU/Linux que simplifique la administración de
todo el documento.

La última sección de la introducción usualmente sí tiene un título estandar que
es ``Objetivos y estructura del documento'', donde se presentan \emph{en prosa}
los objetivos general y específicos que ha tenido el proyecto de tesis,
así como la estructura de la tesis (por ejemplo, ``en el siguiente capítulo se
esbozan los fundamentos teóricos necesarios para explicar en el
\capref{ch:solucion} la propuesta realizada$\ldots$''

%%% Local Variables: 
%%% mode: latex
%%% TeX-master: "main"
%%% End: 
