%% ---------------------------------------------------------------------------
%% intro.tex
%%
%% Introduction
%%
%% $Id: intro.tex 1477 2010-07-28 21:34:43Z palvarado $
%% ---------------------------------------------------------------------------

\chapter{Introducción}
\label{chp:intro}


En la era de la exploración espacial moderna, la implementación de software embebido a bordo de computadoras de guía, navegación y control (GNC) representa un desafío técnico clave para alcanzar la autonomía y precisión que requieren las misiones espaciales. La capacidad de estos sistemas para adaptarse a condiciones variables y entornos hostiles depende en gran medida de flujos de trabajo bien estructurados que garanticen la eficiencia, confiabilidad y seguridad del software embebido. En esta tesis, desarrollada en el laboratorio de sistemas espaciales del Tecnológico de Costa Rica (SETEC-Lab), se presenta un conjunto de flujos de trabajo diseñados específicamente para la implementación de software en computadoras de GNC espaciales. Este enfoque busca optimizar los procesos de desarrollo y validación, asegurando que el software pueda cumplir con los estrictos requerimientos de las misiones y contribuir al éxito de la exploración y operación en el espacio profundo.

\section{Proceso de diseño de los sistemas de Guía, Navegación y Control espacial}

Los sistemas de guía, navegación y control (GNC) son fundamentales en las misiones espaciales, están encargados de determinar la trayectoria óptima para cumplir los objetivos de la misión, además de calcular la secuencia de maniobras necesarias, determinar la posición, velocidad y orientación, también se encargan de aplicar las acciones correctivas necesarias para mantener la trayectoria \cite{hewing2023enhancing}.

La implementación de sistemas GNC en sistemas embebidos, conlleva una combinación de hardware y software especializado, por un lado, los microprocesadores y sistemas en chip son los encargados de gestionar los cálculos, mientras que los sensores proporcionan distintos tipos de datos por medio de las entradas. Por otro lado, los sistemas en tiempo real garantizan la respuesta en el momento requerido. Las aplicaciones de estos se pueden observar en drones, satélites y sondas espaciales \cite{MathWorks}.

\subsection{Requerimientos de los sistemas}

Los requerimientos de los sistemas GNC incluyen: precisión para determinar la posición y orientación del vehículo con gran exactitud, robustez para funcionar de manera confiable y tolerar fallos o perturbaciones generadas por el entorno, autonomía para poder operar sin depender de la intervención humana, flexibilidad para adaptarse a diferentes fases de la misión y un bajo consumo de potencia para minimizar el uso de los recursos limitados a bordo. 
Para cumplir con los requerimientos mencionados anteriormente se debe definir con precisión los requerimientos del sistema, como la precisión necesaria para determinar la posición del vehículo, la robustez del sistema para resistir fallos, las restricciones energéticas y de recursos computacionales a bordo. Una vez solventados estos requerimientos el sistema se plantea bajo una arquitectura modular la cual divide el sistema en bloques independientes para las funciones de guía, navegación y control, facilitando el desarrollo, prueba y mantenimiento \cite{AEM2017}. 


\section{Sistemas embebidos para los sistemas GNC}

El uso de sistemas embebidos ha transformado la navegación y el control aeroespacial. Estos sistemas integran hardware y software, permitiendo el procesamiento de datos en tiempo real, fundamental para la navegación precisa y el control de vuelo. Los sistemas embebidos gestionan sensores que recopilan información sobre altitud, velocidad y posición, permitiendo a pilotos y sistemas automáticos tomar decisiones rápidas y fundamentadas. Esta capacidad de respuesta es esencial en entornos cambiantes, como la aviación o el lanzamiento de cohetes. \cite{Castao2014EstimacinDP}

Además, los sistemas embebidos facilitan la integración de múltiples funciones en un solo dispositivo, reduciendo el peso y volumen de los equipos a bordo, un factor crucial en la industria aeroespacial. Por ejemplo, en los sistemas de control de vuelo, los microcontroladores y procesadores embebidos pueden gestionar desde la navegación hasta la comunicación y el monitoreo de sistemas críticos, todo desde una única unidad. Esta integración mejora la eficiencia del espacio y minimiza la posibilidad de fallos al reducir el número de componentes individuales que podrían fallar. \cite{Culp1993GuidanceAC}

Finalmente, la implementación de sistemas embebidos ha permitido avances significativos en la automatización y la inteligencia artificial aeroespacial. Los algoritmos embebidos procesan datos de manera eficiente, permitiendo la navegación autónoma y el control de vehículos sin intervención humana, especialmente relevante en misiones espaciales con comunicación limitada. Los sistemas embebidos mejoran la seguridad y eficiencia de las operaciones  aeroespaciales, abriendo nuevas posibilidades para la exploración y el desarrollo de tecnologías futuras en este campo. \cite{Falcoz2023GuidanceN}

\subsection{Marco de Trabajo Yocto Project}

Yocto Project es una iniciativa de código abierto que proporciona un conjunto de herramientas y recursos para crear sistemas operativos Linux personalizados, especialmente diseñados para dispositivos embebidos. Su objetivo principal es facilitar el desarrollo de software y la integración de componentes en una amplia variedad de hardware, permitiendo a los desarrolladores construir imágenes de sistema adaptadas a sus necesidades específicas. Utiliza BitBake, una herramienta que permite definir recetas para la construcción y empaquetado de software, lo que otorga gran flexibilidad y personalización. \cite{salvador2014embedded}

Además, soporta múltiples arquitecturas de hardware, como ARM, x86, MIPS y PowerPC, lo que lo hace adecuado para diferentes dispositivos, desde microcontroladores hasta sistemas más complejos. Al fomentar la reutilización de componentes y contar con una comunidad activa que contribuye con mejoras y documentación, el Yocto Project se convierte en una solución ideal para el desarrollo de sistemas operativos en aplicaciones de IoT, electrónica de consumo y automatización industrial \cite{vaduva2015learning}.

\section{Hardware en el loop}
El Hardware en el loop es una técnica fundamental en el desarrollo de sistemas GNC, ya que, permite simular el comportamiento del hardware en tiempo real, facilitando para los desarrolladores la prueba y validación del software sin requerir el hardware físico, es esta forma permite probar de forma exhaustiva el software asegurando el funcionamiento del hardware simulado y permite la validación de todo el sistema antes de su implementación final. Esta implementación genera una mayor precisión en las pruebas, la posibilidad de validar la autonomía del sistema, además de reducir significativamente el tiempo y los costos de implementación y prueba del hardware. En  resumen, es una técnica esencial en el desarrollo de sistemas GNC espaciales, permitiendo una validación integral y eficiente de estos complejos sistemas antes de su despliegue en misiones reales \cite{mihalivc2022hardware} \cite{montoya2020advanced}.




\section{Objetivos y estructura del documento}

El objetivo principal de este proyecto es desarrollar un conjunto de flujos de trabajo para la implementación de software a bordo de computadoras de guía, navegación y control espacial. 

Para lograr este objetivo se persiguen tres objetivos específicos. 

Identificar una plataforma de hardware para el desarrollo de un modelo de ingeniería de una computadora de navegación espacial.

Establecer flujos de trabajo para el prototipado de algoritmos de control de orientación y navegación para aplicaciones espaciales con hardware en el loop. 

Evaluar los casos de uso de una computadora de navegación y control espacial mediante la implementación de una aplicación de referencia demostrativa.

Este documento incluye lo siguiente: en el capítulo 2 se presenta el marco teórico, donde se esbozan los fundamentos de la propuesta realizada. En el capítulo 3 se elige la plataforma de desarrollo para implementar solución propuesta. En el capítulo 4 se presenta el flujo de trabajo desarrollado. En el capítulo 5 se muestran los resultados obtenidos. Por último, el capítulo 6 se presenta las conclusiones de la investigación y trabajo realizado, así como recomendaciones y trabajo a futuro por desarrollar.

%%% Local Variables: 
%%% mode: latex
%%% TeX-master: "main"
%%% End: 
