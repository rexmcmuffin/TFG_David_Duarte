\chapter{Solución propuesta}
\label{ch:especifico2}

Como se pudo obervar en el capitulo \ref{ch:especifico1}, se realizo la seleccion de la tarjeta de desarrollo Zedboard para el desarrollo del proyecto, en este capitulo se pretende establecer flujos de trabajo para el prototipado de algoritmos de control de orientacion y navegacion para aplicaciones espaciales. Esto mediante el uso de Matlab Simulink para tomar un caso de uso como ejemplo, seguido de esto convertir el codigo por medio de la trasnformacion de modelo de simulink a modelo de codigo C, con el objetivo de poder embedder el codigo en una imagen minima por medio del flujo de trabajo de Yocto Project y finalmente probar el mismo en la tarjeta de desarrollo zedboard y asi comparar los resultados obtenidos y el tiempo de ejecucion que llevo la tarea en el computador y en el sistema embebido.

\section{GNC embebido software workflow}

Anteriormente definimos un flujo de trabajo para poder establecer el prototipado de algoritmos de control, orientacion y navegacion para aplicaciones espaciales, el mismo prentendemos logaralo mediante la seleccion de un caso de estudio en matlab simulink el cual se mostrara en el desarrollo de este capitulo

\section{Flujo de trabajo de la aplicación Model 2 Model Transformation}
\subsection{Caso de estudio}
\section{Flijo de Trabajo Herramienta desarrollada por mi persona}
\subsection{Caso de estudio}
\section{Reflexion final}