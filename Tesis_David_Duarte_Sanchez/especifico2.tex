\chapter{Solución propuesta}
\label{ch:especifico2}

Como se pudo obervar en el capitulo \ref{ch:especifico1}, se realizo la seleccion de la tarjeta de desarrollo Zedboard para el desarrollo del proyecto, en este capitulo se pretende establecer flujos de trabajo para el prototipado de algoritmos de control de orientacion y navegacion para aplicaciones espaciales. Esto mediante el uso de Matlab Simulink para tomar un caso de uso como ejemplo, seguido de esto convertir el codigo por medio de la trasnformacion de modelo de simulink a modelo de codigo C, con el objetivo de poder embedder el codigo en una imagen minima por medio del flujo de trabajo de Yocto Project y finalmente probar el mismo en la tarjeta de desarrollo zedboard y asi comparar los resultados obtenidos y el tiempo de ejecucion que llevo la tarea en el computador y en el sistema embebido.

\section{GNC embebido software workflow}

Anteriormente definimos un flujo de trabajo para poder establecer el prototipado de algoritmos de control, orientacion y navegacion para aplicaciones espaciales, el mismo prentendemos logaralo mediante la seleccion de un caso de estudio en matlab simulink el cual se mostrara en el desarrollo de este capitulo

\section{Flujo de trabajo de la aplicación Model 2 Model Transformation}
caso de estudio matlab embedded coder



\subsection{Caso de estudio}


\section{Flijo de Trabajo Herramienta desarrollada por mi persona}
Diagrama de pasos para poder ejecutar el flujo de trabajo

\subsection{Sistema operativo para desarrollo}
hablar sobre el sistema operativo base que se reqiere para generar el flujo, version de kernel y demas daros 

\subsection{Generacion de un contenedor}
por que se debe de utilizar un contenedor para fines del proyecto
\subsubsection{Creacion de un usuario no root}
por que se requiere generar un usuario no root 
\subsection{Yocto Project}
referencar al marco teorico de que es y para que se esta utilizando en el desarrollo del marco de trabajo
\subsection{creacion de una capa de yocto}
como se debe de generar una capa en yocto
\subsection{Caso de estudio}
\subsection{integracion del programa generado a la capa de yocto}
como se debe de intgrar y para que funciona cada comando del .bb de la capa
\subsection{Generacion de la imagen minima}
Por que se usa una imagen minima 
\subsection{Implementacion de la imagen minima en la tarjeta de desarrollo zedboard}
Como se debe de implementar el boot de la tarjeta, como se debe de implementar el  sistema de archivos
\subsection{Conexion de la tarjeta de desarrollo con el computador host}
diagrama de conexion y programas utilizados para llevar a cabo el enlace

\subsection{Ejecucion del caso de estudio y resultados}

\subsection{Comparacion de resultados}

\section{Reflexion final}