%% Este archivo contiene toda la configuración básica del documento de
%% tesis, para centralizar alguna información que se requiere en todo
%% el documento.

%% DRAFT MODE

%%   El modo borrador activa las listas de cosas por hacer, con su
%%   índice, y algunas marcas explícitas de "borrador" por todo lado.
%%
%%   Asegúrse de que esta variable sea false en la versión final y de haber
%%   actualizado la fecha de la defensa, un poco más abajo.
\setboolean{draftmode}{true}            % turn draft mode on
%\setboolean{draftmode}{false}          % turn draft mode off

%% Esta es la fecha que se colocará en el modo borrador
\defDraftDate{\today}
%% Esta es la fecha que se usará en la versión final
\defFinalDate{24 de noviembre, 2023}

%% Este es el nombre del estudiante y el pronombre que utiliza, para cambiar
%% las portadas como corresponde

% Con el nombre de autor, se debe especificar el género a utilizar:
%
%   [M]asculino
%   [F]emenino (usando "señora" donde corresponda)
%   [f]emenino (usando "señorita" donde corresponda)
%   persona [N]o binaria
%
%   Debido a la falta de norma en español para las personas no binarias,
%   posíblemente deba ajustarse para el gusto de cada quien.
%
\defAuthor[M]{David Duarte Sánchez}                % Nombre del estudiante
%\defAuthor[f]{María del Pilar Pérez Prado}    % Nombre de la estudiante

\defAuthorShort{D.~Duarte}                      % Nombre corto
\defAuthorTECID{2017239606}                     % Carné

%% Este es el título completo del informe del trabajo final de graduación.
%% Usted puede agregar \\ para forzar líneas nuevas en la portada y automática-
%% mente el comando se encarga de eliminar eso cuando necesita el título
\defTitle{Diseño de un sistema para soportar algoritmos \\ 
          de control basados en aprendizaje automático \\
          en tiempo real}

%% Palabras clave
\defKeywords{palabras, clave, energía, cambio climático, RISC V}

%% Tribunal Evaluador
%%
%% Indique los nombres de los lectores y asesor
%% El parámetro opcional es
%%  [M]asculino,
%%  [F]emenino,
%%  persona [N]o binaria
\defLectorI[F]{Dra.\,María Curie Pérez}
\defLectorII[M]{M.Sc.\,Pedro Pérez Pereira}
\defAsesor[M]{Ing.\,Albert Einstein Sánchez}

%% Tipo de tesis o informe
%%   - Tesis de Licenciatura
%%   - Informe de Proyecto Final 
%%   - Tesis de Maestría
\defTFGType{Tesis de Licenciatura}

%% Nombre del departamento e institución
\defInstitution{Instituto Tecnológico de Costa Rica}
\defDepartment{Escuela de Ingeniería Electrónica}

