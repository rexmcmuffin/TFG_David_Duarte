\documentclass{article}
\usepackage[utf8]{inputenc}
\usepackage{geometry}
\usepackage{booktabs}
\usepackage{longtable}
\usepackage[spanish]{babel}
\geometry{
    a4paper,
    left=20mm,
    right=20mm,
    top=20mm,
    bottom=20mm,
}

\title{Proyecto: Diseño de un sistema para soportar algoritmos de control basados en aprendizaje automático en tiempo real}
\author{David Felipe Duarte Sánchez}
\date{\today}

\begin{document}

\maketitle
\section*{Bitácora Semana 1}

\begin{longtable}[c]{|c|c|c|p{6cm}|}
\hline
\textbf{Fecha} & \textbf{Horas} & \textbf{Actividad} & \textbf{Descripción} \\
\hline
\endhead
\hline
\endfoot
07/02/2024 & 5 & Corrección Anteproyecto & Cambio en los objetivos generales y específicos del proyecto \\
08/02/2024 & 8 & Investigación para satisfacer los nuevos objetivos & Investigación  sobre los sistemas en tiempo real y la relación que tienen con los sistemas de control  automático, revision de \cite{alonso2010panoramica} \cite{zhao2017hardware} y \cite{kim1994software}\\
09/02/2024 & 7 & Redacción y revisión del documento & Se redactó de nuevo el documento escrito y se presentó por medio de TEC DIgital \\
\hline
\end{longtable}


\begin{itemize}
    \item Cambio de objetivo general y específico.
    \item Incorporación de los indicadores y entregables
    \item Planteamiento de las  alternativas
    \item Cambio en la seccion de Generalidades
    \item Cambio en la seccion de Entorno del proyecto
\end{itemize}

Lista de referencias consultadas para el entorno del proyecto, las generalidadesy  las alternativas propuestas


\nocite{de2000introduccion}
\nocite{alonso2010panoramica}
\nocite{munoz1994extensiones}
\nocite{zhao2017hardware}
\nocite{shi2011vcuda}
\nocite{deng2020model}
\nocite{duato2011enabling}.
\nocite{scarpino2011opencl} 
\nocite{owaida2011massively}
\nocite{meyer2020evaluating} 
\nocite{verma2016accelerating}
\nocite{galan2000control} 
\nocite{kim1994software} 
\nocite{razdan1994prisc}
\nocite{owaida2011massively}

\bibliography{../Borrador_Anteproyecto_Duarte_Sanchez_David_Felipe/biblio.bib}
\bibliographystyle{plain}

\newpage
\section*{Bitácora Semana 2}

\begin{longtable}[c]{|c|c|c|p{6cm}|}
    \hline
    \textbf{Fecha} & \textbf{Horas} & \textbf{Actividad} & \textbf{Descripción} \\
    \hline
    \endhead
    \hline
    \endfoot
    13/02/2024 & 5 & Busqueda de Material bibliográfico &  Referencias bibliográficas relacionadas con control automático, sistemas en tiempo real y aceleración por hardware con cuda, open cl y one api. \\
    \hline
\end{longtable}
    

    %\bibliography{../Borrador_Anteproyecto_Duarte_Sanchez_David_Felipe/biblio.bib}
    %\bibliographystyle{plain}


%Investigar como se encha a andar un sistema entiempo real 

%buena investigación a nivel de los otros  sistemas de tiempo
%
%como se usan en sistemas de contrl 
%
%buscar pappers en sitios importantes
%
%
%
%linux  real time / 
%
%
%https://www.intel.com/content/www/us/en/developer/tools/oneapi/training/academic-program.html#gs.4zp3le 
%
%buscar ago parecido con opencl con cuda y demas ,,,,,
%
%OpenCL
%CUDA
%
%diferencias entre open cl y one API (principales diferencias entre las cosas, buscar en white pappers)
%
%
%investigacion con objetivo de buscar que es lo que yo voy a inmplementar, es para fundamentar y argumnetar el por que de lo que voy a hacer, entonces toda la infroacion sirve para comparar y demas para de esa forma tomar una mejor desicion 
%
%en este abito en el cual nos movemos puede ser muy util buscar tesis de grado en otros lados de gente que impemente plantas de control  y esta referencia a otras y asi es mas facil la revision bibliografica 
%
%https://repository.tudelft.nl/islandora/object/uuid:baab4d3d-7d55-496e-a82d-cfddf75aa01d/datastream/OBJ/download
%https://webthesis.biblio.polito.it/27644/1/tesi.pdf
%
%buscar pasos de nvidia en cuanto a implementacion para agragar a la comparacion de las referencias bibliograficas 

%se habla de iniciar una fuerte investigacion bibliografca con el objetivo de generar datos para la redaccion del marco teorico del proyecto en cuestion.
%
%Motores de busqueda para la investigacion de los temas definidos 
    



    
\begin{itemize}
    \item Buscar training de cuda, open cl, one api, para conocer más de estos sistemas.
    \item Por medio de la revisión bibliográfica lograr definir que es lo que se va a implementar en el proyecto y fundamentar el porqué lo estoy aplicando
    \item Búsqueda de referencias bibliográficas en tesis doctorales para conocer más sobre la implementación de plantas de control
    \item Redacción del marco teórico del proyecto
\end{itemize}
    

\end{document}