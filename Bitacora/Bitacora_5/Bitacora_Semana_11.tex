\documentclass[12pt,letterpaper]{article}
\usepackage{listings}
\usepackage{graphicx}
\usepackage[table,xcdraw]{xcolor}
\RequirePackage{xcolor}
\definecolor{tecAzul}{cmyk}{1,0.91,0.33,0.25} % según manual de imagen 2016
\definecolor{tecRojo}{cmyk}{0,0.9,0.86,0}     % según manual de imagen 2016

\renewcommand{\familydefault}{\sfdefault}
\usepackage{amsmath} % for the equation* environment
\usepackage{mwe}
\usepackage{graphicx}
\usepackage[spanish]{babel}
\usepackage{multirow}
\usepackage{titlesec}
\titleformat*{\section}%
{\normalfont\Large\bfseries\color{tecAzul}}
\titleformat*{\subsection}%
{\normalfont\large\bfseries\color{tecAzul}}


\usepackage[tmargin=2cm,bmargin=2cm,lmargin=2.5cm,rmargin=2.5cm]{geometry}
\usepackage{textpos}
\usepackage{tikz}
\usepackage{pgfplots}
\usepackage{pgf}

\usepackage[margin=1cm]{caption}

\usepackage{hyperref}

%
% paragraph layout
%
\parindent0em                           % indentation width of first line
\parskip1.3ex                           % space between paragraphs


\newcommand{\EstudianteA}{David F. Duarte Sánchez}

\pgfplotsset{compat=1.17}



\usepackage{listings}
\usepackage{xcolor}

\definecolor{codegreen}{rgb}{0,0.6,0}
\definecolor{codegray}{rgb}{0.5,0.5,0.5}
\definecolor{codepurple}{rgb}{0.58,0,0.82}
\definecolor{backcolour}{rgb}{0.95,0.95,0.92}

\lstdefinestyle{mystyle}{
    backgroundcolor=\color{backcolour},   
    commentstyle=\color{codegreen},
    keywordstyle=\color{magenta},
    numberstyle=\tiny\color{codegray},
    stringstyle=\color{codepurple},
    basicstyle=\ttfamily\footnotesize,
    breakatwhitespace=false,         
    breaklines=true,                 
    captionpos=b,                    
    keepspaces=true,                 
    numbers=left,                    
    numbersep=5pt,                  
    showspaces=false,                
    showstringspaces=false,
    showtabs=false,                  
    tabsize=2
}

\lstset{style=mystyle}



\begin{document}
	
\graphicspath{{./}{./fig/}}

%-------------------------- Title section -------------------------------------%

%
\begin{textblock}{10}[0,0](-0.5,0)
	\large Escuela de Ingeniería Electrónica \\ 
	EL5617 Trabajo Final de Graduación \\
\end{textblock}

%
\begin{textblock}{10}[0,0](2.6,-0.35)
	\begin{flushright}
		\includegraphics[scale=0.8]{Firma_TEC-4.pdf}
	\end{flushright}
\end{textblock}

%% Title %%
\begin{center}
	\vspace{70mm}
	{\large\color{tecRojo} Trabajo Final de Graduación}
	\par\vspace{8mm}
	{\Large\bf\color{tecAzul}{Bitácora de Trabajo - Entrega 6}}
	\par\vspace{100mm}
	{{\EstudianteA \\ II Semestre 2024} 
	\vspace{8mm}}
\end{center}

\newpage
%------------------------------------------------------------------------------%

\renewcommand{\baselinestretch}{1.1}    % line spacing

%------------------------------------------------------------------------------%

\section{Semana 10}
\subsection{Corrección de Tesis}

\bf{Fecha de trabajo:} 27/09/2024.\\
\bf{Objetivo:} Compilación y generación del caso de uso por medio del flujo de trabajo propuesto.

\begin{table}[h!]
    \resizebox{\textwidth}{!}{%
    \begin{tabular}{|l|}
        \hline
        \hline
        \multicolumn{1}{|c|}{Reporte de   actividades} \\ \hline
        \hline
        -Compilación cruzada del programa C generado por medio del flujo de trabajo de Matlab Simulink \\
        haciendo uso de la herramienta Simulink Coder. \\ \hline

        -Creación de una capa personalizada en yocto para la integración del archivo binario compilado por \\
        medio del método de compilación cruzada.\\ \hline

        -Documentación del capítulo 4 con los pasos específicos que se siguieron desde la generación del caso \\
        de uso en simulink como la compilación e implementación del mismo en Linux, pasando por el compilador \\
        de arm en Linux y la generación de la capa. \\\hline

        -Avance en el trabajo escrito\\\hline
    \end{tabular}}
\end{table}

  \begin{figure}[h!]
    \centering
    \includegraphics[width=0.8\textwidth]{images/diagrama3.png}
    \caption{Flujo Simulink Coder}
    \label{fig:Simulink_coder}
  \end{figure}

%\bibliography{bibliografia_consultada}
%\bibliographystyle{plain}
\end{document}